\documentclass{article}
\usepackage[utf8]{inputenc}
\usepackage{amsmath}

\title{BAW Model and Extensions}
\author{Frank Ma}
\date{November 2015}

\begin{document}

\maketitle

\begin{abstract}
    This paper demonstrates an analytical approximation of American style option solutions named as the BAW model.
    Derivations and extensions are provided in details.
\end{abstract}


\section{Introduction}

While European style option is analytically priced by Black-Scholes-Merton model, American style option is usually valued numerically. 
Barone-Adesi and Whaley (1987) demonstrated an approximation approach to evaluate American style options semi-analytically. 
The method starts from the intuitive fact that any American option must be no cheaper than the European option with same term and strike.
\begin{equation}
    V_{Ame} = V_{Eur} + V_{prem}
\end{equation}
Where $ V_{prem} \geq 0 $, and all three components in above must be governed by the general Black-Scholes-Merton model Partial Differential Equation as below 
\begin{equation}
    \frac{\partial V}{\partial t} + \left(r - q\right) S \frac{\partial V}{\partial S} + \frac{1}{2} \sigma^2 S^2 \frac{\partial^2 V}{\partial S^2} - r V = 0
\end{equation}

BAW model then focuses on the early exercise premium $ V_{prem} $ term and manages to find an close enough approximation of PDE which is semi-analytically solvable.
The model finally searches for critical spot price $ S^* $ at which point, option is optimum to be exercised immediately to max out the benefit to the option holder.
\begin{equation} \label{BAW formula}
    V_{Ame} = 
    \begin{cases} 
        V_{Eur} + V_{prem}, & \mbox{if $ \eta S < \eta S^* $} \\
        \eta \left(S - K\right), & \mbox{if $ \eta S \geq \eta S^* $}
    \end{cases}
\end{equation}
Where $ \eta $ is a binary variable with value of $ 1 $ for call option and $ -1 $ for put.


\section{Derivation}

We start from applying a few change of variables to simplify further derivations.
Define time to maturity as $ \tau = T - t$, early exercise premium as $ V_{prem} = g\left(\tau\right) \cdot h\left(S, g\left(\tau\right)\right) $, and $ g\left(\tau\right) = 1 - e^{-r \tau} $.
The substitute into the BSM PDE to find the following equation.
\begin{equation}
    -\frac{\partial g}{\partial \tau} h - g \frac{\partial h}{\partial g} \frac{\partial g}{\partial \tau} + (r - q) S g \frac{\partial h}{\partial S} + \frac{1}{2} \sigma^2 S^2 g \frac{\partial^2 g}{\partial S^2} - r g h = 0
\end{equation}

Knowing the first derivative of $ g $ function is $ g' = r e^{-r \tau} = r \left(1 - g\right) $, we reformat the above PDE into below.
\begin{equation}
    -r \left(1 - g\right) h - g \frac{\partial h}{\partial g} r \left(1 - g\right) + (r - g) S g \frac{\partial h}{\partial S} + \frac{1}{2} \sigma^2 S^2 g \frac{\partial^2 h}{\partial S^2} - r g h = 0
\end{equation}

Reorganize the PDE with respect to $ S $ to find
\begin{equation}
    \frac{1}{2} \sigma^2 S^2 g \frac{\partial^2 h}{\partial S} + \left(r - q\right) S g \frac{\partial h}{\partial S} - r h - g r \left(1 - g\right) \frac{\partial h}{\partial \tau} = 0
\end{equation}

Note the last term $ g r \left(1 - g\right) \frac{\partial h}{\partial \tau} $ is equivalent to $ g \frac{\partial h}{\partial \tau} $ by chain rule.
Barone-Adesi and Whaley \cite{Efficient Analytic Approximation of American Option Values} approximated this term as zero to reduced the PDE into an analytically solvable ODE of $ h $ with respect to $ S $ as bellow.

\begin{equation}
    S^2 \frac{\partial^2 h}{\partial S^2} + S \frac{2 \left(r - q\right)}{\sigma^2} \frac{\partial h}{\partial S} - \frac{2 r h}{\sigma^2 g} = 0
\end{equation}

For clarity, define two coefficients $ \beta_{1} = \frac{2 \left(r - q\right)}{\sigma^2} $ and $ \beta_{0} = \frac{2 r}{\sigma^2} $ and rewrite as below.
\begin{equation}
    S^2 \frac{\partial^2 h}{\partial S} + S \beta_{1} \frac{\partial h}{\partial S} - \frac{\beta_0}{g} h = 0
\end{equation}

Assume function $ h $ is in the form of $ \alpha S^{\lambda} $, then the first and the second derivative with respect to $ S $ are $ \alpha \lambda S^{\lambda - 1} $ and $ \alpha \lambda \left(\lambda - 1\right) S^{\lambda - 2} $.
The above ODE is simplified as below.
\begin{equation}
    \alpha S^{\lambda} \left(\lambda^2 + \left(\beta_{1} - 1\right) \lambda - \frac{\beta_{0}}{g} \right) = 0
\end{equation}
    
Two solutions of $ \lambda $ can be found.
\begin{equation}
    \lambda_{0, 1} = \frac{-\left(\beta_{1} - 1\right) \mp \sqrt{\left(\beta_{1} - 1\right)^2 + 4 \frac{\beta_{0}}{g}}}{2}
\end{equation}

Since the ODE is in second order, the general solution must be in the format of the following.
\begin{equation}
    h = \alpha_{0} S^{\lambda_{0}} + \alpha_{1} S^{\lambda{1}}
\end{equation}

The yet to be solved coefficients are $ \alpha_{0} $ and $ \alpha_{1} $ which need to be implied through boundaries analysis.
Assuming strictly positive risk-free rate, $ \frac{\beta_{0}}{g} > 0 $ thus $ \lambda_{0} < 0 $ and $ \lambda_{1} > 0$.
These features indicates that at extreme values of $ S $, function $ h $ could grow exponentially as $ \alpha_{0} S^{\lambda_{0}} \to \infty $ when $ S \to 0 $ , or $ \alpha_{1} S^{\lambda_{1}} \to \infty $ when $ S \to \infty $.
This apparently contradicts to the valuation principal of American option that derivative could not be more expensive than its underlying.
Therefore, for call option, $ \alpha_{0} $ must equals to zero; and for put option, $ \alpha_{1} $ must equals to zero.
To reconcile call and put pricing routines, we reintroduce $ \eta $ to define function $ \alpha(\eta) $ and $ \lambda(\eta) $.
The American option price is organized as below.
\begin{equation}
    V_{Ame} = 
    \begin{cases}
        V_{Eur} + g \alpha S^{\lambda}, & \mbox{if $ \eta S < \eta S^* $}\\
        \eta (S - K), & \mbox{if $ \eta S \geq \eta S^* $}
    \end{cases}
\end{equation}
Where $ S^* $ is the tipping point that it just turns economical to exercised the option as of valuation time.
At the optimum exercising point, pricing equality must hold so as to the slops of prices.
\begin{align}
    \eta (S^* - K) &= V_{Eur}(S^*) + g \alpha {S^*}^{\lambda} \\
    \eta &= \Delta_{Eur}(S^*) + g \alpha \lambda {S^*}^{\lambda - 1}
\end{align}

Then the $ \alpha $ term is implied as $ \alpha = \frac{\eta - \Delta_{Eur}(S^*)}{g \lambda {S^*}^{\lambda - 1}} $ and the final task is to find the optimum exercise spot $ S^* $.
A closed-form solution of $ S^* $ may not be easily worked out, but straightforward solvers such as Newton-Rahpson method can find the proper value with high precision and efficiency.
Let us define a function $ f(S) $ with minimization target of zero.
\begin{equation}
    f(S) = \eta (S - K) - V_{Eur}(S) - \frac{\left(\eta - \Delta_{Eur}(S)\right)  S}{\lambda}
\end{equation}

The first derivative against $ S $ is as below.
\begin{equation}
    f'(S) = \left(\eta - \Delta_{Eur}(S)\right) \left(1 - \frac{1}{\lambda}\right) + \frac{\Gamma_{Eur}(S) S}{\lambda}
\end{equation}

Given a reasonable initial guess of $ S^0 $, a closer value to $ S^{*} $ is implied through Newton-Raphson method which progresses recursively until $ \left|f(S)\right| $ is within desired tolerance level.
\begin{equation}
    S^{i + 1} = S^i - \frac{f(S^i)}{f'(S^i)}
\end{equation}

With the last piece estimating optimum exercise edge full-filled, American style option price is approximated as the following.
\begin{equation}
    V_{Ame} = 
    \begin{cases} 
        V_{Eur}\left(S\right) + \frac{\left(\eta - \Delta_{Eur}(S^*)\right)}{\lambda} S^*  {\left(\frac{S}{S^*}\right)}^{\lambda}, & \mbox{if $ \eta S < \eta S^* $} \\
        \eta \left(S - K\right), & \mbox{if $ \eta S \geq \eta S^* $}
    \end{cases}
\end{equation}
Where $ \lambda = \frac{-\left(\beta_{1} - 1\right) + \eta \sqrt{\left(\beta_{1} - 1\right)^2 + 4 \frac{\beta_{0}}{g}}}{2} $, $ g = 1 - e^{-r \tau} $, $ \beta_{0} = \frac{2 r}{\sigma^2} $, and $ \beta_{1} = \frac{2 \left(r - q\right)}{\sigma^2} $.


\section{Greeks}

Compared to BSM model, BAW model is a semi-analytic formula which introduces a  complexity on implying closed form formulas for Greek calculation.
The optimum exercising spot $ S^* $ is not a function of underlying $ S $, therefore Delta and Gamma should be relatively straightforward to work out.

\subsection{Delta}
Delta $ \frac{\partial V}{\partial S} $ is the first order sensitivity of option price against the movement of underlying.
Straightforward derivation of the pricing function leads to the following.
\begin{equation}
    \Delta_{Ame} =
    \begin{cases}
        \Delta_{Eur} + \left(\eta - \Delta\left(S^*\right)\right) \left(\frac{S}{S^*}\right)^{\lambda - 1} , & \mbox{if $ \eta S < \eta S^* $} \\
        \eta, & \mbox{if $ \eta S \geq \eta S^* $} 
    \end{cases}
\end{equation}

\subsection{Gamma}
Delta $ \frac{\partial^2 V}{\partial S^2} $ defines the second order sensitivity against underlying $ S $, or the first order sensitivity against the $ \Delta $ term.
A further step derivative of Delta against underlying finds formula in below.
\begin{equation}
    \Gamma_{Ame} = 
    \begin{cases}
        \Gamma_{Eur} + \left(\eta - \Delta\left(S^*\right)\right) \frac{\lambda - 1}{S^*} \left(\frac{S}{S^*}\right)^{\lambda - 2}, & \mbox{if $ \eta S < \eta S^* $} \\
        0, & \mbox{if $ \eta S \geq \eta S^* $} 
    \end{cases}
\end{equation}

\subsection{Theta}
Theta $ \frac{\partial V}{\partial t} $ measures the first order sensitivity of option price with respect to the time decay.
In the pricing formula, multiple components $ g, \lambda, S^*, \text{and } \Delta $ are a function of time.
This makes the analytical formula too complex to be useful.
Indirect approach is adopt since the general BSM PDE must hold under all circumstances.
\begin{equation}
    \Theta = r V - \left(r - q\right) S \Delta - \frac{1}{2} \sigma^2 S^2 \Gamma
\end{equation}


\section{Extensions}

Barone-Adesi and Whaley reduced the PDE by ignoring the term $ g r \left(1 - g\right) \frac{\partial h}{\partial g} $.
This assumption is only appropriate for short and long terms as numerical studies indicate the term structure of this term is actually none-zero with a peak at mid-term.

Endeavours are made to overcome the inaccuracy of BAW model in mid-term expiries.
Ju and Zhong (1999) introduced a correction term to the $ h $ function as $ h = h_1 + h_2 $ with $ h_2 = \varepsilon h_1 $.


\begin{thebibliography}{9}
    \bibitem{Efficient Analytic Approximation of American Option Values}
    Barone-Adesi G., Whaley R.
    \textit{Efficient Analytic Approximation of American Option Values}
    The Journal of Finance, June, 1987.
    
    \bibitem{An Approximated Formula for Pricing American Options}
    Ju N., Zhong R.
    \textit{An Approximated Formula for Pricing American Options}
    Journal of Derivatives, Winter, 1999.
\end{thebibliography}

\end{document}
