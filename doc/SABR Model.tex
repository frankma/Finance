\documentclass{article}
\usepackage[utf8]{inputenc}
\usepackage{amsmath}

\title{SABR Model}
\author{Frank Ma}
\date{Initial Creation: July 2015 \\ Last Update: April 2016}

\begin{document}

\maketitle

\begin{abstract}
    This paper introduces SABR model and its related applications.
    A few special cases are studied and the model calibration is documented for references.
\end{abstract}

\section{Introduction}

Stochastic Alpha Beta and Rho (SABR) model is a stochastic volatility model managing to capture the volatility smiles.
The underlying term is modeled with an exponential parameter $ \beta $ while the stochastic volatility term is modeled as a Log-Normal process which naturally avoids negative values.
The two process are correlated with a coefficient $ \rho $.
\begin{align}
    d F &= \sigma F^{\beta} d W^F \\
    d \sigma &= \nu \sigma d W^{\sigma} \\
    \text{where } & d W^F d W^{\sigma} = \rho d t \text{, and } {\sigma}_0 = \alpha \text{.}\nonumber
\end{align}

Practitioners usually find two primary uses of SABR model: volatility surface interpolations and sensitivity measurements.
This parametrized model not only produces smooth volatility smile but also provides sensible implied volatilities when the forward is shocked.
A crucial drawback of the model, however, is the lacking of evolution from one expiry to the next hence restrained it from path dependent instruments pricing.

In financial markets, quote is usually referenced as implied volatility in addition to its price.
To match the market quotes in the volatility space, Hagan et al \cite{Managing Smile Risk} present a few semi-analytic approximations.
The following two sections examine the approximations and their applications closely.


\section{Log-normal Volatility Approximation}

Black 76 or Black-Scholes-Merton model assumes underlying as a log-normal process.
A semi-analytic approximation of log-normal volatility is given as below.
\begin{subequations}
    \begin{align}
        \sigma_B{(F, K)} &\approx \frac{\alpha}{(FK)^{\frac{1 - \beta}{2}}\left(1 + \frac{\left(1 - \beta\right)^2}{24} \ln^2{\frac{F}{K}} + \frac{{\left(1 - \beta\right)}^{4}}{1920} \ln^4{\frac{F}{K}}\right)} \cdot \frac{\zeta}{\chi (\zeta)} \nonumber \\
        & \quad \cdot \left(1 + \left(\frac{\alpha^2 {\left(1 - \beta\right)}^{2}}{24 \left(F K\right)^{1 - \beta}} + \frac{\alpha \beta \nu \rho}{4 {\left(F K\right)}^{\frac{1 - \beta}{2}}} + \frac{\nu^2 \left(2 - 3 \rho^2\right)}{24}\right) T\right)\\
        \zeta &= \frac{\nu}{\alpha} {(F K)}^{\frac{1 - \beta}{2}} \ln{\frac{F}{K}} \\
        \chi(\zeta) &= \ln{\frac{\sqrt{1 - 2 \rho \zeta + {\zeta}^2} - \rho + \zeta}{1 - \rho}}
    \end{align}
\end{subequations}

Under the special case when $ \beta = 1 $, the underlying is modeled as Log-Normal process and the approximation could be reduced to the below.
\begin{subequations} \label{SABR_LogNormal_Beta_1}
    \begin{align}
        \sigma_B^{\beta = 1}{(F, K)} &= \alpha \frac{\zeta}{\chi(\zeta)} \left(1 + \left(\frac{\alpha \nu \rho}{4} + \frac{\nu^2 \left(2 - 3 \rho^2\right)}{24}\right) T \right) \\
        \zeta &= \frac{\nu}{\alpha} \ln{\frac{F}{K}} \\
        \chi(\zeta) &= \ln{\frac{\sqrt{1 - 2 \rho \zeta + \zeta^2} - \rho + \zeta}{1 - \rho}}
    \end{align}
\end{subequations}

Note, $ \zeta $ is zero for ATM quotes, so as to $ \chi(\zeta) $.
L'H\^opital's rule resolves the undefined point of $ \frac{\zeta}{\chi(\zeta)} $ in the ATM case as below.
\begin{align}
    \lim_{\zeta \to 0}{\chi(\zeta)} &= \lim_{\zeta \to 0}{\frac{1}{\frac{1 - \rho}{\left(1 - 2 \rho \zeta + \zeta^2\right)^{\frac{1}{2}} - \rho + \zeta} \cdot \frac{\frac{1}{2}\left(1 - 2 \rho \zeta + \zeta^2\right)^{-\frac{1}{2}}\left(-2\rho + 2 \zeta \right) + 1}{1 - \rho}}} \nonumber \\
    &= \lim_{\zeta \to 0}{\frac{\left(1 - 2 \rho \zeta + \zeta^2\right)^{\frac{1}{2}}}{\frac{1 - \rho}{\left(1 - 2 \rho \zeta + \zeta^2\right)^{\frac{1}{2}} - \rho + \zeta} \cdot \frac{-\rho + \zeta + \left(1 - 2 \rho \zeta + \zeta^2\right)^{\frac{1}{2}}}{1 - \rho}}} \nonumber \\
    &= \lim_{\zeta \to 0}{\left(1 - 2 \rho \zeta + \zeta^2\right)^{\frac{1}{2}}} \nonumber \\
    &= 1
\end{align}


\section{Normal Volatility Approximation}

Normal or Bachelier model assumes underlying follows normal process.
The normal volatility of underlying can be approximated as below.
\begin{subequations}
    \begin{align}
        \sigma_N{(F, K)} &\approx \alpha {\left(F K\right)}^{\frac{\beta}{2}} \frac{1 + \frac{1}{24} \ln^2{\frac{F}{K}} + \frac{1}{1920} \ln^4{\frac{F}{K}}}{1 + \frac{\left(1 - \beta\right)^{2}}{24} \ln^2{\frac{F}{K}} + \frac{\left(1 - \beta\right)^{4}}{1920} \ln^4{\frac{F}{K}}} \cdot \frac{\zeta}{\chi \left(\zeta\right)}\nonumber \\
        & \quad \cdot \left(1 + \left(-\frac{\alpha \beta (2 - \beta)}{24 {\left(F K\right)}^{1 - \beta}} + \frac{\alpha \beta \nu \rho}{4 {\left(F K\right)}^{\frac{1 - \beta}{2}}} + \frac{\nu^2 \left(2 - 3 \rho^2\right)}{24}\right) T \right)\\
        \zeta &= \frac{\nu}{\alpha} {(F K)}^{\frac{1 - \beta}{2}} \ln{\frac{F}{K}} \\
        \chi(\zeta) &= \ln{\frac{\sqrt{1 - 2 \rho \zeta + {\zeta}^2} - \rho + \zeta}{1 - \rho}}
    \end{align}
\end{subequations}

Under special case when $ \beta = 0 $, the underlying is modeled as Normal process and the approximation could be reduced to the below.
\begin{subequations}
    \begin{align}
        \sigma_N^{\beta=0} &\approx \alpha \cdot \frac{\zeta}{\chi(\zeta)} \cdot \left(1 + \frac{\nu^2 \left(2 - 3 \rho^2\right)}{24} T \right) \\
        \zeta &= \frac{\nu}{\alpha} (F K)^{\frac{1}{2}} \ln{\frac{F}{K}} \\
        \chi(\zeta) &= \ln{\frac{\sqrt{1 - 2 \rho \zeta + \zeta^2} - \rho + \zeta}{1 - \rho}}
    \end{align}
\end{subequations}

Practically, when $ \beta $ is set to zero and normal volatility is requested, none-positive forward or strike is expected.
In the case of zero or negative forward or strike, $ \zeta $ term is undefined.
Therefore, the volatility requires a different and meaningful approximation as below.
\begin{equation}
    \zeta = \frac{\nu}{\alpha} \left(F - K\right)
\end{equation}

\section{Sensitivities}



\section{Local Volatility}

Local variance
\begin{equation} \label{Local Vol General Form}
    \sigma_{L}^2 = \frac{\sigma_{B}^2 + 2 \sigma_{B} T \left(\frac{\partial \sigma_{B}}{\partial T} + (r_T - q_T) K \frac{\partial \sigma_{B}}{\partial K}\right)}{\left(1 - \frac{K \ln{\frac{K}{F}}}{\sigma_{B}} \frac{\partial \sigma_{B}}{\partial K}\right)^2 + K \sigma_{B} T \left(\frac{\partial \sigma_{B}}{\partial K} - \frac{1}{4} K \sigma_{B} T \left(\frac{\partial \sigma_{B}}{\partial K}\right)^2 + K \frac{\partial^2 \sigma_{B}}{\partial K^2}\right)}
\end{equation}

When $ \beta $ equals to one, the log-normal volatility approximation collapses to a relative simple structure as shown in formula \ref{SABR_LogNormal_Beta_1}.
The reduced form greatly simplifies the analytical expression of the local volatility formula \ref{Local Vol General Form}.
Before deriving the derivatives of black volatility against strike, a few intermediate derivatives are listed as below.
\begin{subequations}
    \begin{align}
        \frac{d \zeta}{d K} &= -\frac{\nu}{\alpha K} \\
        \frac{d^2 \zeta}{d K^2} &= \frac{\nu}{\alpha K^2}\\
        \frac{d \chi}{d \zeta} &= \left(1 - 2 \rho \zeta + \zeta^2\right)^{-\frac{1}{2}}\\
        \frac{d^2 \chi}{d \zeta^2} &= -\left(1 - 2 \rho \zeta + \zeta^2\right)^{-\frac{3}{2}} \left(\zeta - \rho\right)
    \end{align}
\end{subequations}

Further, define function $ G = \frac{\zeta}{\chi\left(\zeta\right)} $ and $ S = \left(1 - 2 \rho \zeta + \zeta^2\right)^{\frac{1}{2}} $ and find the derivatives of function $ G $ against $ \zeta $ as the following.
\begin{subequations}
    \begin{align}
        \frac{d S}{d \zeta} &= S^{-1} \left(\zeta - \rho\right) \\
        \frac{d G}{d \zeta} &= \frac{\chi S - \zeta}{\chi^2 S} \\
        \frac{d^2 G}{d \zeta^2} &= \frac{\chi^2 S \left(\frac{1}{S} S + \chi \frac{1}{S} \left(\zeta - \rho\right) - 1\right) - \left(\chi S - \zeta\right) \left(2 \chi \frac{1}{S} S + \chi^2 \frac{1}{S} \left(\zeta - \rho\right)\right)}{\chi^4 S^2} \nonumber \\
        &= \frac{-2 \chi S^2 + 2 S \zeta + \zeta \chi \left(\zeta - \rho\right)}{\chi^3 S^3} \nonumber \\
        &= \frac{-2 \chi \left(1 - 2\rho \zeta + \zeta^2\right) + 2 S \zeta + \zeta^2 \chi - \rho \chi \zeta}{\chi^3 S^3} \nonumber \\
        &= \frac{\left(-\zeta^2 + 3 \rho \zeta - 2\right) \chi + 2 S \zeta}{\chi^3 S^3}
    \end{align}
\end{subequations}

The derivatives of black volatility against ttm and strike when $ \beta $ equals to one are as below.
\begin{subequations}
    \begin{align}
        \frac{d \sigma_{B}^{\beta=1}}{\partial T} &= \alpha \frac{\zeta}{\chi \left(\zeta\right)} \left(\frac{\alpha \rho \nu}{4} + \frac{\nu^2 \left(2 - 3 \rho^2\right)}{24}\right) \label{dSigdT} \\
        \frac{d \sigma_{B}^{\beta=1}}{\partial K} &= -\frac{\nu}{K} \frac{\partial G}{\partial \zeta} \left(1 + \left(\frac{\alpha \rho \nu}{4} + \frac{\nu^2 \left(2 - 3 \rho^2\right)}{24}\right) T\right) \label{dSigdK} \\
        \frac{d^2 \sigma_{B}^{\beta=1}}{d K^2} &= \left(\frac{\nu}{K^2} \frac{\partial G}{\partial K} - \frac{\nu}{\alpha K^2} \frac{\partial^2 G}{\partial K^2}\right) \left(1 + \left(\frac{\alpha \rho \nu}{4} + \frac{\nu^2 \left(2 - 3 \rho^2\right)}{24}\right) T\right) \label{dSig2dK2}
    \end{align}
\end{subequations}

Same issue persists in the derivatives of $ G $ function for ATM volatilities where $ \zeta $ is zero.
To overcome the problem, Taylor expansions are applied to find close but stable approximations.
The following formulas expand till the third order.
\begin{subequations}
    \begin{align}
        G^{K \approx F} &= 1 - \frac{1}{2} \rho \zeta + \left(-\frac{1}{4} \rho^2 + \frac{1}{6}\right) \zeta^2 \\
        \frac{d G^{K \approx F}}{d K} &= -\frac{1}{2} \rho  + 2 \left(-\frac{1}{4} \rho^2 + \frac{1}{6}\right) \zeta - \frac{1}{8} \left(6 \rho^2 - 5\right) \rho \zeta^2 \\
        \frac{d^2 G^{K \approx F}}{d K^2} &= 2 \left(-\frac{1}{4} \rho^2 + \frac{1}{6}\right) - \frac{1}{4} \left(6 \rho^2 - 5\right) \rho \zeta + 12 \left(-\frac{5}{16} \rho^4 + \frac{1}{3} \rho^2 - \frac{17}{360}\right) \zeta^2
    \end{align}
\end{subequations}


\section{Analytic Density Function}

Independent from modeling, density function $ \mathcal{P}\left(T, S_T\right) $ equals to the second order derivative of pre-discounted option price $ V^T $ against strike $ K $.
\begin{equation}
    \mathcal{P}\left(T, S_T\right) = \frac{\partial^2 V^T}{\partial K^2}
\end{equation}

Define Black 76 model forward price function $ V^T $ as $ U\left(T, K, \sigma(T, K)\right) $.
Then chain rule finds the density function as below.
\begin{align}
    \frac{d U}{d K} &= \frac{\partial U}{\partial K} + \frac{\partial U}{\partial \sigma} \frac{\partial \sigma}{\partial K} \\
    \frac{d^2 U}{d K^2} &= \frac{\partial^2 U}{\partial K^2} + \frac{\partial^2 U}{\partial K \partial \sigma} \frac{\partial \sigma}{\partial K}  + \frac{\partial^2 U}{\partial \sigma \partial K} \frac{\partial \sigma}{\partial K} + \frac{\partial^2 U}{\partial \sigma^2} \left(\frac{\partial \sigma}{\partial K}\right)^2 + \frac{\partial U}{\partial \sigma} \frac{\partial^2 \sigma}{\partial K^2} \nonumber \\
    &= \frac{\partial^2 U}{\partial K^2} + 2 \frac{\partial^2 U}{\partial K \partial \sigma} \frac{\partial \sigma}{\partial K} + \frac{\partial^2 U}{\partial \sigma^2} \left(\frac{\partial \sigma}{\partial K}\right)^2 + \frac{\partial U}{\partial \sigma} \frac{\partial^2 \sigma}{\partial K^2} \label{d2U_per_dK2}
\end{align}

In above formula \ref{d2U_per_dK2} of the forward density function, terms $ \frac{\partial U}{\partial \sigma} $, $ \frac{\partial^2 U}{\partial \sigma^2} $, $ \frac{\partial^2 U}{\partial K \partial \sigma} $ and $ \frac{\partial^2 U}{\partial K^2} $ correspond to pre-discounted black formula greeks $ Vega^T $, $ Vomma^T $, $ Vanna_{K}^T $ and $ Gamma_{K}^T $.
\begin{subequations}
    \begin{align}
        \frac{d U}{d \sigma} &= F_{t}^{T} \phi(d_1) \sqrt{\tau} \\
        \frac{d^2 U}{d \sigma^2} &= F_{t}^{T} \phi(d_1) \sqrt{\tau} \cdot \frac{d_1 d_2}{\sigma} \\
        \frac{d^2 U}{d K d \sigma} &= F_{t}^{T} \phi(d_1) \cdot \frac{d_1}{K \sigma} \\
        \frac{d^2 U}{d K^2} &= \frac{\phi(d_2)}{K \sigma \sqrt{\tau}}
    \end{align}
\end{subequations}

Two terms on the derivative of black volatility against strike $ \frac{d \sigma}{d K} $ and $ \frac{d^2 \sigma}{d K^2} $ could be implied from SABR model.
In the reduced form of log-normal volatility approximation as formula \ref{SABR_LogNormal_Beta_1}, derivatives of black implied volatility against strike $ \frac{d \sigma}{d K} $ and $ \frac{d^2 \sigma}{d K^2} $ can be found in formula \ref{dSigdK} and \ref{dSig2dK2}.


\section{Parameter Estimation and Model Calibration}

For any given expiry, the model has four parameters need to be calibrated.
From model structure and parameters experimentation, we find strong


\begin{thebibliography}{9}
    \bibitem{Managing Smile Risk}
    Hagan, P., Kumar, D., Lesniewski, A., Woodward, D.
    \textit{Managing Smile Risk}
    Wilmott Magazine, September, 2002.
    
    \bibitem{The SABR Model}
    Rouah, F.
    \textit{The SABR Model}
    www.FRouah.com
    
    \bibitem{Derivation of Local Volatility}
    Rouah, F.
    \textit{Derivation of Local Volatility}
    www.FRouah.com
\end{thebibliography}

\end{document}
