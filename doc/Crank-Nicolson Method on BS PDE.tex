\documentclass{article}
\usepackage[utf8]{inputenc}
\usepackage{amsmath}

\title{Crank-Nicolson Method on BS PDE}
\author{Frank Ma}
\date{October 2015}

\begin{document}

\maketitle

\begin{abstract}
    This paper describes details on the Crank-Nicolson method to discretize the partial differential equation of Black-Scholes-Merton model.
\end{abstract}


\section{Introduction}

Black-Scholes Equation lays the cornerstone for the modern quantitative finance.
It starts from defining a stochastic process of the underlying asset git sas formula \ref{BS SDE} and then a generic derivative value function $ V(S, t) $.
Later the Partial Differential Equation \ref{BS PDE} is provided to govern the evolution of derivative value.
\begin{align}
    & d S = \mu S d t + \sigma S d W \label{BS SDE} \\
    & \frac{\partial V}{\partial t} + r S \frac{\partial V}{\partial S} + \frac{1}{2} \sigma^2 S^2 \frac{\partial^2 V}{\partial S^2} - r V = 0 \label{BS PDE}
\end{align}

To solve present value of derivative prices, multiple approaches such as Analytic, Monte Carlo Simulation and Finite Difference can be used.
In this paper, we discuss Finite Difference methods with focus on Crank-Nicolson.


\section{Crank-Nicolson Method}

Crank-Nicolson method is a discretization tool to solve heat equations numerically.
This method assures error term converges in second order with respect to both time and space as it is based on trapezoidal rule.
We arbitrary define two consecutive time steps of three spatially adjacent points as below.
\begin{equation}
    \begin{array}{|l c|l}
        V_{t - \Delta t}^{S - \Delta S} & \quad & V_{t}^{S - \Delta S} \\
        V_{t - \Delta t}^{S} & \quad & V_{t}^{S} \\
        V_{t - \Delta t}^{S + \Delta S} & \quad & V_{t}^{S + \Delta S}
    \end{array}
\end{equation}

Specific to the Black-Scholes model, PDE is solved backward. The LHS state will be in $ t - \Delta t $ while the RHS one will be in $ t $.
A generic discretization can be found as follows.
\begin{align}
    & \frac{V_{t - \Delta t}^{S} - V_{t}^{S}}{-\Delta t} + r S \frac{1}{2} \left(\frac{V_{t - \Delta t}^{S + \Delta S} - V_{t - \Delta t}^{S - \Delta S}}{2 \Delta S} + \frac{V_{t}^{S + \Delta S} - V_{t}^{S - \Delta S}}{2 \Delta S}\right) \nonumber \\
    & \quad + \frac{1}{2} \sigma^2 S^2 \frac{1}{2} \left(\frac{V_{t - \Delta t}^{S + \Delta S} - 2 V_{t - \Delta t}^{S} + V_{t - \Delta t}^{S - \Delta S}}{{(\Delta S)}^2} + \frac{V_{t}^{S + \Delta S} - 2 V_{t}^{S} + V_{t}^{S - \Delta S}}{{(\Delta S)}^2} \right) \nonumber \\
    & \quad - r \frac{1}{2} \left(V_{t - \Delta t}^{S} + V_{t}^{S}\right) = 0 \label{BS PDE CN Discretized}
\end{align}

Reorganize the left-hand-side of the equation in the state of $ t - \Delta t $ while the right-hand-side in the state of $ t $.
\begin{align}
    & \left(\frac{r S}{4 \Delta S} - \frac{\sigma^2 S^2}{4 (\Delta S)^2)}\right) V_{t - \Delta t}^{S - \Delta S} + \left(\frac{1}{\Delta t}  + \frac{\sigma^2 S^2}{2 (\Delta S)^2} + \frac{r}{2}\right) V_{t - \Delta t}^{S} \nonumber \\
    & \quad + \left(-\frac{r S}{4 \Delta S} - \frac{\sigma^2 S^2}{4 (\Delta S)^2}\right) V_{t - \Delta t}^{S + \Delta S} = \left(-\frac{r S}{4 \Delta S} + \frac{\sigma^2 S^2}{4 (\Delta S)^2}\right) V_{t}^{S - \Delta S} \nonumber \\
    & \quad + \left(\frac{1}{\Delta t} - \frac{\sigma^2 S^2}{2 (\Delta S)^2} - \frac{r}{2}\right) V_{t}^{S} + \left(\frac{r S}{4 \Delta S} + \frac{\sigma^2 S^2}{4 (\Delta S)^2}\right) V_{t}^{S + \Delta S}
\end{align}

In general, the discretized PDE is transformed into a linear system as below.
\begin{equation}
    {LHS} \cdot V_{t - \Delta t} = {RHS} \cdot V_{t}
\end{equation}


\section{Boundary Conditions}

Special care is needed on the boundaries of the solution domain as the central limit discretization is undefined on the edge.
There are a few standard approaches to overcome the boundary issue:
Dirichlet boundary condition which specifies fixed values on the boundaries;
or Neumann boundary condition that provides deterministic first order derivative on the boundaries. 

In this paper we presents a method similar to Neumann method by creating phantom out-of-bound points through Linear extrapolation.
This application also results in zero second derivative on the bounds $ \frac{\partial^2 V}{\partial S^2} = 0 $ as linear extrapolation is surely curve-less.

At the lower bound, immediate out bound point is $ V^{-1} = 2 V^{0} - V^{1} $ and the central limit discretization on the bound will be $ \frac{V^{1} - 2 V^{0} - V^{1}}{2 \Delta S} = \frac{V^{1} - V^{0}}{\Delta S}$. The discretization of PDE on the lower bound will be as follows.
\begin{align}
    & \frac{V_{t - \Delta t}^{S^{0}} - V_{t}^{S^{0}}}{-\Delta t} + r S^{0} \frac{1}{2} \left(\frac{V_{t - \Delta t}^{S^{1}} - V_{t - \Delta t}^{S^{0}}}{\Delta S} + \frac{V_{t}^{S^{1}} - V_{t}^{S^{0}}}{\Delta S}\right) \nonumber \\
    & \quad - r \frac{1}{2} \left(V_{t - \Delta t}^{S^{0}} + V_{t}^{S^{0}}\right) = 0 \nonumber \\
    & \left(\frac{1}{\Delta t} + \frac{r S^{0}}{2 \Delta S} + \frac{r}{2}\right) V_{t - \Delta t}^{S^{0}} + \left(-\frac{r S^{0}}{2 \Delta S}\right) V_{t - \Delta t}^{S^{1}} \nonumber \\
    & \quad = \left(\frac{1}{\Delta t} - \frac{r S^{0}}{2 \Delta S} - \frac{r}{2}\right) V_{t}^{S^{0}} + \left(\frac{r S^{0}}{2 \Delta S}\right) V_{t}^{S^{1}}
\end{align}

At the upper bound, immediate out bound point is $ V^{n + 1} = 2 V^{n} - V^{n - 1} $ and the central limit discretization on the bound will be $ \frac{2 V^{n} - V^{n - 1} - V^{n - 1}}{2 \Delta S} = \frac{V^{n} - V^{n - 1}}{\Delta S}$.
The discretization of PDE on the upper bound will be as follows.
\begin{align}
    & \frac{V_{t - \Delta t}^{S^{n}} - V_{t}^{S^{n}}}{-\Delta t} + r S^{n} \frac{1}{2} \left(\frac{V_{t - \Delta t}^{S^{n}} - V_{t - \Delta t}^{S^{n - 1}}}{\Delta S} + \frac{V_{t}^{S^{n}} - V_{t}^{S_{n-1}}}{\Delta S}\right) \nonumber \\
    & \quad -r \frac{1}{2} \left(V_{t - \Delta t}^{S^{n}} + V_{t}^{S^{n}}\right) = 0 \nonumber \\
    & \left(\frac{r S^{n}}{2 \Delta S}\right) V_{t - \Delta t}^{S^{n - 1}} + \left(\frac{1}{\Delta t} - \frac{r S^{n}}{2 \Delta S} + \frac{r}{2}\right) V_{t - \Delta t}^{S^{n}} \nonumber \\
    & \quad = \left(-\frac{r S^{n}}{2 \Delta S}\right) V_{t}^{S^{n - 1}} + \left(\frac{1}{\Delta t} + \frac{r S^{n}}{\Delta S} - \frac{r}{2}\right) V_{t}^{S^{n}}
\end{align}

With the boundary condition absorbed into the transition matrix, we re-organize tri-diagonal matrices as follows.
\begin{align}
    LHS &= \left| \begin{matrix}
        1 + \frac{r S^{0} \Delta t}{2 \Delta S} + \frac{r \Delta t}{2} & - \frac{r S^{0} \Delta t}{2 \Delta S} & 0 \\
        \vdots & \ddots & \vdots \\
        \frac{r S^* \Delta t}{4 \Delta S} - \frac{\sigma^2 {S^*}^2 \Delta t}{4 (\Delta S)^2} & 1 + \frac{\sigma^2 {S^*}^2 \Delta t}{2 (\Delta S)^2} + \frac{r \Delta t}{2} & -\frac{r S^* \Delta t}{4 \Delta S} - \frac{\sigma^2 {S^*}^2 \Delta t}{4 (\Delta S)^2} \\
        \vdots & \ddots & \vdots \\
        0 & -\frac{r S^{n} \Delta t}{\Delta S} & 1 - \frac{r S^{n} \Delta t}{2 \Delta S} + \frac{r \Delta t}{2}
    \end{matrix} \right| \\ \nonumber \\
    RHS &= \left| \begin{matrix}
        1 - \frac{r S^{0} \Delta t}{2 \Delta S} - \frac{r \Delta t}{2} & \frac{r S^{0} \Delta t}{2 \Delta S} & 0 \\
        \vdots & \ddots & \vdots \\
        - \frac{r S^* \Delta t}{4 \Delta S} + \frac{\sigma^2 {S^*}^2 \Delta t}{4 (\Delta S)^2} & 1 - \frac{\sigma^2 {S^*}^2 \Delta t}{2 (\Delta S)^2} - \frac{r \Delta t}{2} & \frac{r S^* \Delta t}{4 \Delta S} + \frac{\sigma^2 {S^*}^2 \Delta t}{4 (\Delta S)^2}\\
        \vdots & \ddots & \vdots \\
        0 & \frac{r S^{n} \Delta t}{\Delta S} & 1 + \frac{r S^{n} \Delta t}{2 \Delta S} - \frac{r \Delta t}{2}
    \end{matrix} \right|
\end{align}

The evolution of the derivative price vector is as follows.
\begin{align}
    LHS \left| \begin{matrix} V_{t - \Delta t}^{S^0} \\ \vdots \\ V_{t - \Delta t}^{S^*} \\ \vdots \\ V_{t - \Delta t}^{S^n} \end{matrix}\right| &= RHS \left| \begin{matrix} V_{t}^{S^0} \\ \vdots \\ V_{t}^{S^*} \\ \vdots \\ V_{t}^{S^n} \end{matrix}\right| \\
    LHS \cdot V_{t - \Delta t} &= RHS \cdot V_{t}
\end{align}

The finite difference solving process is the the governed by the following system.
\begin{equation}
    V_{t - \Delta t} = {LHS}^{-1} \cdot RHS \cdot V_{t} \label{BS PDE discretized}
\end{equation}

When the PDE is Homogeneous, formula \ref{BS PDE discretized} can be transformed for engineering benefit as below
\begin{equation}
    V_{t - \Delta t}' = V_{t}' \cdot RHS' \cdot {LHS'}^{-1}
\end{equation}


\section{Convergence Speed Analysis}

One major benefit from the Crank-Nicolson method is the second order convergence rate on the the time discretization.
To prove it, we apply Taylor expansion to $ V_{t - \Delta t}^{*} $, $ V_{*}^{S - \Delta S} $ and $ V_{*}^{S + \Delta S} $ till second order with respect to time and third order with respect to space.
\begin{align}
    V_{t - \Delta t}^{*} &= V_{t}^{*} - \frac{\partial V_{t}^{*}}{\partial t} \Delta t + \frac{1}{2} \frac{\partial^2 V_{t}^{*}}{\partial t^2} (\Delta t)^2 + \mathcal{O}\left(\left(\Delta t\right)^3\right) \\
    V_{*}^{S - \Delta S} &= V_{*}^{S} - \frac{\partial V_{*}^{S}}{\partial S} \Delta S + \frac{1}{2} \frac{\partial^2 V_{*}^{S}}{\partial S^2} (\Delta S)^2 - \frac{1}{6} \frac{\partial^3 V_{*}^{S}}{\partial S^3} (\Delta S)^3 + \mathcal{O}\left(\left(\Delta S\right)^4\right) \\
    V_{*}^{S + \Delta S} &= V_{*}^{S} + \frac{\partial V_{*}^{S}}{\partial S} \Delta S + \frac{1}{2} \frac{\partial^2 V_{*}^{S}}{\partial S^2} (\Delta S)^2 + \frac{1}{6} \frac{\partial^3 V_{*}^{S}}{\partial S^3} (\Delta S)^3 + \mathcal{O}\left(\left(\Delta S\right)^4\right)
\end{align}

Plug above ones into derivative discretizations $ \frac{V_{t - \Delta t}^{*} - V_{t}^{*}}{-\Delta t} $,  $ \frac{V_{*}^{S + \Delta S} - V_{*}^{S - \Delta S}}{2 \Delta S} $ and $ \frac{V_{*}^{S + \Delta S} - 2 V_{*}^{S} + V_{*}^{S - \Delta S}}{(\Delta S)^2} $ to find the follow relationships, omitting any residual terms higher than the order of two.
\begin{align}
    \frac{V_{t - \Delta t}^{*} - V_{t}^{*}}{-\Delta t} &= \frac{\partial V_{t}^{*}}{\partial t} - \frac{1}{2} \frac{\partial^2 V_{t}^{*}}{\partial t^2} \Delta t + \mathcal{O}\left(\left(\Delta t\right)^2\right) \\
    \frac{V_{*}^{S + \Delta S} - V_{*}^{S - \Delta S}}{2 \Delta S} &= \frac{\partial V_{*}^{S}}{\partial S} + \mathcal{O}\left(\left(\Delta S\right)^2\right) \\
    \frac{V_{*}^{S + \Delta S} - 2 V_{*}^{S} + V_{*}^{S - \Delta S}}{(\Delta S)^2} &= \frac{\partial^2 V_{*}^{S}}{\partial S^2} + \mathcal{O}\left(\left(\Delta S\right)^2\right)
\end{align}

Plug discretization into formula \ref{BS PDE CN Discretized} to get the following equation.
\begin{align}
    & \frac{\partial V_{t}^{S}}{\partial t} - \frac{1}{2} \frac{\partial^2 V_{t}^{S}}{\partial t^2} \Delta t + \mathcal{O}\left(\left(\Delta t\right)^2\right) + r S \frac{1}{2} \left(\frac{\partial V_{t - \Delta t}^{S}}{\partial S} + \frac{\partial V_{t}^{S}}{\partial S} + \mathcal{O}\left(\left(\Delta S\right)^2\right)\right) \nonumber \\
    & \quad + \frac{1}{2} \sigma^2 S^2 \left(\frac{\partial^2 V_{t - \Delta t}^{S}}{\partial S^2} + \frac{\partial^2 V_{t}^{S}}{\partial S^2} + \mathcal{O}\left(\left(\Delta S\right)^2\right)\right) \nonumber \\
    & \quad - r S \frac{1}{2} \left(V_{t - \Delta t}^{S} + V_{t}^{S}\right) = 0 \label{BS PDE CN reverted}
\end{align}

Since base point of is $ V_{t}^{S} $, further discretization of derivatives on time axis is applied as below.
\begin{align}
    \frac{\partial V_{t - \Delta t}^{S}}{\partial S} + \mathcal{O}\left(\left(\Delta S\right)^2\right) &= \frac{\partial V_{t}^{S}}{\partial S} - \frac{\partial \frac{\partial V_{t}^{S}}{\partial S}}{\partial t} \Delta t + \mathcal{O}\left(\left(\Delta t\right)^2, \left(\Delta S\right)^2\right) \\
    \frac{\partial^2 V_{t - \Delta t}^{S}}{\partial S^2} + \mathcal{O}\left(\left(\Delta S\right)^2\right) &= \frac{\partial^2 V_{t}^{S}}{\partial S^2} - \frac{\partial \frac{\partial^2 V_{t}^{S}}{\partial S^2}}{\partial t} \Delta t + \mathcal{O}\left(\left(\Delta t\right)^2, \left(\Delta S\right)^2\right)
\end{align}

With further time discretization on derivatives, formula \ref{BS PDE CN reverted} transforms into the following.
\begin{align}
    & \frac{\partial V_{t}^{S}}{\partial t} - \frac{1}{2} \frac{\partial^2 V_{t}^{S}}{\partial t^2} \Delta t + r S \frac{1}{2} \left(\frac{\partial V_{t}^{S}}{\partial S} - \frac{\partial \frac{\partial V_{t}^{S}}{\partial S}}{\partial t} \Delta t + \frac{\partial V_{t}^{S}}{\partial S}\right) \nonumber \\
    & \quad + \frac{1}{2} \sigma^2 S^2 \frac{1}{2} \left(\frac{\partial^2 V_{t}^{S}}{\partial S^2} - \frac{\partial \frac{\partial^2 V_{t}^{S}}{\partial S^2}}{\partial t} \Delta t + \frac{\partial^2 V_{t}^{S}}{\partial S^2}\right) \nonumber \\
    & \quad -r \frac{1}{2} \left(V_{t}^{S} - \frac{\partial V_{t}^{S}}{\partial t} \Delta t + V_{t}^{S}\right) + \mathcal{O}\left(\left(\Delta t\right)^2, \left(\Delta S\right)^2\right) = 0 \nonumber \\
    & \frac{\partial V_{t}^{S}}{\partial t} + r S \frac{\partial V_{t}^{S}}{\partial S} + \frac{1}{2} \sigma^2 S^2 \frac{\partial^2 V_{t}^{S}}{\partial S^2} -r V_{t}^{S} + \mathcal{O}\left(\left(\Delta t\right)^2, \left(\Delta S\right)^2\right) \nonumber \\
    & \quad - \frac{1}{2} \left(\frac{\partial^2 V_{t}^{S}}{\partial t^2} + r S \frac{\partial \frac{\partial V_{t}^{S}}{\partial S}}{\partial t} + \frac{1}{2} \sigma^2 S^2 \frac{\partial \frac{\partial^2 V_{t}^{S}}{\partial S^2}}{\partial t} - r \frac{\partial V_{t}^{S}}{\partial t} \right) \Delta t = 0 \label{BS PDE CN reverted 2}
\end{align}

Because the expression $ \frac{\partial^2 V_{t}^{S}}{\partial t^2} + r S \frac{\partial \frac{\partial V_{t}^{S}}{\partial S}}{\partial t} + \frac{1}{2} \sigma^2 S^2 \frac{\partial \frac{\partial^2 V_{t}^{S}}{\partial S^2}}{\partial t} - r \frac{\partial V_{t}^{S}}{\partial t} $ is indeed the first derivative of the left-hand-side Black-Scholes PDE with respect to $ t $ and this term must be zero, equation \ref{BS PDE CN Discretized} end up to be equivalent to Black Scholes PDE with error term of $ \mathcal{O}\left(\left(\Delta t\right)^2, \left(\Delta S\right)^2\right) $.

\end{document}