\documentclass{article}
\usepackage[utf8]{inputenc}
\usepackage{amsmath}

\title{Bounded Transformation}
\author{Frank Ma}
\date{September 2015}

\begin{document}

\maketitle

\section{Introduction}

When numerically solving optimization problems, bounded variables are usually expected while most optimization process assumes unbounded independent variables.
Variable projection become quite important for such situations.
The key point is to create a bijective function with one end bounded while the other is unbounded.


\section{Absolute Function}

\begin{subequations}
\begin{align}
    y &= l + \frac{u - l}{2} \left(1 + \frac{x}{1 + \left|x\right|}\right) \\
    x &= \begin{cases}
            1 - \frac{1}{2} \frac{u - l}{y - l}, \quad & y < \frac{l + u}{2} \\
            -1 - \frac{1}{2} \frac{u - l}{y - u}, \quad & y \geq \frac{l + u}{2}
        \end{cases}
\end{align}
\end{subequations}


\section{Arc-tangent Function}

\begin{subequations}
\begin{align}
    y &= l + \left(\frac{\arctan(x)}{\pi} + \frac{1}{2}\right) \left(u - l \right) \\
    x &= \tan\left(\frac{y - l}{u - l}\pi - \frac{\pi}{2}\right)
\end{align}
\end{subequations}


\section{Normal Function}

\begin{subequations}
\begin{align}
    y &= l + \left(u - l\right) \Phi(x) \\
    x &= \frac{\Phi^{-1}(y - l)}{u - l}
\end{align}
\end{subequations}


\end{document}
