\documentclass{article}
\usepackage[utf8]{inputenc}
\usepackage{amsmath}

\title{Volatility Smile Extrapolation Method}
\author{Frank Ma}
\date{November 2015}

\begin{document}

\maketitle

\begin{abstract}
    This paper explores a few methods to extrapolate volatility smile.
    No-arbitrage and curve smoothness are critically concerned.
\end{abstract}


\section{Introduction}

Quotes of volatility instrument are finite, usually scattered sparsely.
Great amount of applications requires off-node market quotes which could never be available to the practitioners.
Interpolation and extrapolation techniques are used to find such points uncommon in the liquid market but critical to applications. 

Extrapolation methods are the focus in this paper. 
he study starts from the constrains of volatility surface and move on to the extrapolation methods.
The conclusion make proper comparison of the introduced methods.


\section{Constrains}

\subsection{Risk Neutral Forward Density}
Forward value of the option prices $ C_T^T $ or $ P_T^T $ at expiry $ T $ can be written as an expectation of the following.
\begin{subequations}
    \begin{align}
        C_T^T &= \int_{K}^{\infty}{(S_T - K) \mathcal{P}\left(S_T\right)} d S_T \\
        P_T^T &= \int_{0}^{K}{(K - S_T) \mathcal{P}\left(S_T\right)} d S_T
    \end{align}
\end{subequations}
Where $ \mathcal{P}(S_T) $ defines the forward density function of $ S_T $.

Taking the derivative of option forward prices against strikes to reach the probability cumulative function.
Note, the Leibniz Integral Rule as
\begin{equation}
    \frac{d \int_{a(t)}^{b(t)}{f(x, t)} d x}{d t} = \int_{a(t)}^{b(t)}{\frac{\partial f(x, t)}{\partial t} d x} + f\left(b(t), t\right) \frac{d b(t)}{d t} - f\left(a(t), t\right) \frac{d a(t)}{d t}
\end{equation}

\begin{subequations}
    \begin{align}
        \frac{\partial C_T^T}{\partial K} &= -\int_{K}^{\infty}{\mathcal{P}(S_T) d S_T} + 0 - (K - K) \mathcal{P}(S_T) \cdot 1\nonumber \\
        &= -\int_{K}^{\infty}{\mathcal{P}(S_T) d S_T} \\
        \frac{\partial P_T^T}{\partial K} &= \int_{0}^{K}{\mathcal{P}(S_T) d S_T} + \left(K - K\right) \mathcal{P}(S_T) \cdot 1 - 0\nonumber \\
        &= \int_{0}^{K}{\mathcal{P}(S_T) d S_T}
    \end{align}
\end{subequations}

Furthermore, second derivative of option forward price against strike reaches probability density function as bellow

\begin{subequations}
    \begin{align}
        \frac{\partial^2 C_T^T}{\partial K^2} &= -0 + 0 + \mathcal{P}(K) \cdot 1 \nonumber \\
        &= \mathcal{P}(K) \\
        \frac{\partial^2 P_T^T}{\partial K^2} &= 0 + \mathcal{P}(K) \cdot 1 - 0 \nonumber \\
        &= \mathcal{P}(K)
    \end{align}
\end{subequations}

Apparently, valid value ranges of Cumulative-Density-Function (CDF) and Probability-Density-Function (PDF) are $ [0, 1] $ and $ [0, +\infty) $.
That means, second derivative of forward option price against strike $ \frac{\partial^2 V}{\partial K^2} $ must be strictly positive to assure arbitrage-free volatility quotes. 

\subsection{Smoothness}
Smoothness of the volatility surface is also a reasonable concern of volatility surface construction as exotic instrument pricing usually rely on a robust and smooth volatility surface. 


\section{Polynomial Method}

Benaim, Dodgson and Kainth (2008) described a method to extrapolate volatility surface on the price domain as following.
\begin{subequations}
    \begin{align}
        P &= K^{\lambda} \cdot e^{\beta_0 + \beta_1 K + \beta_2 K^2} \\
        C &= K^{-\lambda} \cdot e^{\beta_0 + \beta_1 K^{-1} + \beta_2 K^{-2}}
    \end{align}
\end{subequations}

To reconcile the above formula, we change variable from $ K $ to $ X $ with the relationship of $ X = K^{-\eta} $, where $ \eta = 1 $ for out-the-money call options, and $ \eta = -1 $ for out-the-money put options.
OTM options are used because their prices converge to zero with deeper out-the-money.
$ \lambda $ term is carefully chosen to enforce the convergence and arbitrage-free.
For low strikes, $ \lambda \geq 1 $ and for high strikes, $ \lambda \geq 0 $.
\begin{subequations}
    \begin{align}   
        V &= X^{\lambda} \cdot e^{\beta_0 + \beta_1 X + \beta_2 X^2} \\
        X &= K^{-\eta} \text{, where } \eta = 
        \begin{cases}
            1 \text{, if call} \\
            -1 \text{, if put}
        \end{cases}
    \end{align}
\end{subequations}

First derivative $ \Delta_X $ against $ X $ can be find as bellow.
\begin{align}
    \frac{\partial V}{\partial X} &= \lambda X^{\lambda - 1} e^{\beta_0 + \beta_1 X + \beta_2 X^2} + X^{\lambda} e^{\beta_0 + \beta_1 X + \beta_2 X^2} \left(\beta_1 + 2 \beta_2 X \right) \nonumber \\
    &= \frac{\lambda}{X} V + V \left(\beta_1 + 2 \beta_2 X\right) \nonumber \\
    &= V \left(\frac{\lambda}{X} + \beta_1 + 2 \beta_2 X\right)
\end{align}

Second derivative $ \Gamma_X $ against $ X $ is as the follwoing.
\begin{align}
    \frac{\partial^2 V}{\partial X^2} &= \frac{\partial V}{\partial X} \left(\frac{\lambda}{X} + \beta_1 + \beta_2 X^2\right) + V \left(-\frac{\lambda}{X^2} + 2 \beta_2\right) \nonumber \\
    &= \frac{\partial V}{\partial X} \frac{\frac{\partial V}{\partial X}}{V} + V \left(-\frac{\lambda}{X^2} + 2 \beta_2\right) \nonumber \\
    &= V \left(\left(\frac{\Delta_k}{V}\right)^2 -\frac{\lambda}{X^2} + 2 \beta_2 \right)
\end{align}

Bijection can be found between the coefficient set of $ \beta_0 $, $ \beta_1 $, and $ \beta_2 $ and the derivative set of $ V $, $ \Delta_X $ and, $ \Gamma_X $.
\begin{subequations}
    \begin{align}
        \beta_2 &= \frac{1}{2} \left(\frac{\lambda}{X^2} + \frac{\Gamma_X}{V} - \left(\frac{\Delta_X}{V}\right)^2\right) \\
        \beta_1 &= \frac{\Delta_X}{V} - \frac{\lambda}{X} - 2 \beta_2 X \nonumber \\
        &= - 2 \frac{\lambda}{X} + \frac{\Delta_X}{V} - \left(\frac{\Gamma_X}{V} - \left(\frac{\Delta_X}{V}\right)^2\right) X \\
        \beta_0 &= \ln{V} - \lambda \ln{X} - \beta_1 X - \beta_2 X^2 \nonumber \\
        &= \ln{V} - \lambda \ln{X} + \frac{3}{2} \lambda - \frac{\Delta_X}{V} X + \frac{1}{2} \left(\frac{\Gamma_X}{V} - \left(\frac{\Delta_X}{V}\right)^2\right) X^2
    \end{align}
\end{subequations}

The natural base is on strike $ K $ not the variable of $ X $.
Apparently, chain rule can be found the following relationships.
\begin{subequations}
    \begin{align}
        \frac{\partial V}{\partial K} &= \frac{\partial V}{\partial X} \frac{\partial X}{\partial K} \\
        \frac{\partial^2 V}{\partial K^2} &= \frac{\partial^2 V}{\partial X^2} \left(\frac{\partial X}{\partial K}\right)^2 + \frac{\partial V}{\partial X} \frac{\partial^2 X}{\partial K^2}
    \end{align}
\end{subequations}

Interchangeably, derivative against strike $ K $ can be transformed into derivative against the substitute variable $ X $ as the following.
\begin{subequations}
    \begin{align}
        \frac{\partial V}{\partial X} &= \frac{\partial V}{\partial K} \bigg/ \frac{\partial X}{\partial K} \\
        \frac{\partial^2 V}{\partial X^2} &= \left(\frac{\partial^2 V}{\partial K^2} - \frac{\partial V}{\partial X} \frac{\partial^2 X}{\partial K^2}\right) \bigg/ \left(\frac{\partial X}{\partial K}\right)^2
    \end{align}
\end{subequations}


Where first and second derivative of $ X $ against K can be found with the cases of $ \eta $ as belows.
\begin{subequations}
    \begin{align}
        \frac{\partial X}{\partial K} &= 
        \begin{cases}
            -K^{-2}, &\eta = 1 \\
            1, &\eta = -1
        \end{cases}\\
        \frac{\partial^2 X}{\partial K^2} &= 
        \begin{cases}
            2 K^{-3}, &\eta = 1 \\
            0, &\eta = -1
        \end{cases}
    \end{align}
\end{subequations}


\section{Conclusion}

Conclusion


\begin{thebibliography}{9}
    \bibitem{An arbitrage-free method for smile extrapolation} 
    Benaim S., Dodgson M., and Kainth D.
    \textit{An arbitrage-free method for smile extrapolation}
    Technical report, Royal Bank of Scotland, 2008
    
    \bibitem{Smile Interpolation and Extrapolation}
    Iwashita Y.
    \textit{Smile Interpolation and Extrapolation}
    OpenGamma Quantitative Research n. 25, 2014
\end{thebibliography}

\end{document}
