\documentclass{article}
\usepackage[utf8]{inputenc}
\usepackage{amsmath}

\title{Black76, BSM and Related Topics}
\author{Frank Ma}
\date{March 2015}


\begin{document}

\maketitle
%  \delimitershortfall=-1pt  % this is for the parentheses size adjustments
 
\begin{abstract}
This document describes Black 76 model with derivation of pricing formula and basic Greeks.
\end{abstract}


\section{Introduction}

As a variant of Black-Scholes-Merton model, Black76 models on forward $ F_t^T $ instead of spot $ S_t $.
The underlying forward has the same expiry as the option; therefore benefits us to focus on the forward only and to bypass the drift term.
The apparent economic implication is no exchange of assets during the delta hedging process.
The dynamics of forward with expiry $ T $ is governed by the following SDE.
\begin{equation}\label{Black76 SDE}
    \frac{1}{F_t^T}  dF_t^T = \sigma dW_t
\end{equation}

Like Black-Scholes-Merton, Black76 provides a theoretical estimation of European option price.
Noticed that the forward at expiry $ F_T^T $ converges to $ S_T $, we formulate the European options payoff as the following.
\begin{equation}\label{Option Payoff}
    \left(\eta \cdot (F_T^T - K) \right)^+, \quad
    \eta =
    \begin{cases} 
        1, & \mbox{if call} \\
        -1, & \mbox{if put}
    \end{cases}
\end{equation}


\section{Valuation}

To price an European option through no-arbitrage theory, we set up a portfolio of option and forwards.
During the life of the option, we apply dynamic delta hedging to re-balance the portfolio.
At expiry $ T $, we end up with a portfolio of option and terminal delta shares of forward.
The overall portfolio value is a deterministic amount.
Equivalently, the hedged portfolio has no differences from a risk less bond.
Additionally, no cash flows is expected during the life of the option, the mix of option and forward is considered as self-financing.
Apply change of variable from $ F_t^T $ to $ \ln{F_t^T} $ and Ito's Lemma finds the new governing SDE of $ \ln{F_t^T} $ as below.
\begin{equation}\label{Black76 ln F SDE}
    d\ln{F_t^T} = -\frac{1}{2} \sigma^2 dt + \sigma dW_t
\end{equation}

Define the time to expiry $ \tau = T - t$ and integrate from current state $ t $ to terminal state $ T $.
\begin{align}\label{Black76 ln F integral}
    \int_{t}^{T} d\ln{F_s^T} &= \int_{t}^{T} -\frac{1}{2} \sigma^2 ds + \int_{t}^{T} \sigma \varepsilon \sqrt{ds} \nonumber\\
    \ln{F_T^T} - \ln{F_t^T} &= -\frac{1}{2} \sigma^2 \tau + \sigma \varepsilon \sqrt{\tau} \nonumber\\
    F_T^T &= F_t^T e^{-\frac{1}{2} \sigma^2 \tau + \sigma  \varepsilon \sqrt{\tau}}
\end{align}

In equation \ref{Black76 ln F integral}, we find the expression of a stock price at option expiry $ T $.
Noticed the only randomness is $ \varepsilon $ which follows a standard normal distribution $ \phi $, we apply change of variable from underlying forward to the standard normal randomness $ \varepsilon $.
\begin{align}\label{Black76 epsilon interval}
    \varepsilon &= \frac{\ln{\frac{F_T^T}{F_t^T}} + \frac{1}{2} \sigma^2}{\sigma \sqrt{\tau}} \nonumber \\
    \varepsilon &\in
    \begin{cases}
        [d, \infty), & \mbox{if call} \\
        (-\infty, d], & \mbox{if put}
    \end{cases}, \quad
    d = \frac{\ln{\frac{K}{F_t^T}} + \frac{1}{2} \sigma^2 \tau}{\sigma \sqrt{\tau}}
\end{align}

Define $ \mathcal{P} (F_T^T) $ as the probability density function of the terminal forward price.
Also by definition, a standard normal distribution has probability density function as $ \phi(x) = \frac{1}{\sqrt{2 \pi}} e^{-\frac{1}{2} x^2} $.
For a call option, the expected option price at expiry $ T $ can be written as the following. 
\begin{align}\label{Call option price derivation}
    \mathrm{E} \left[c_T^T \right] &= \int_{K}^{\infty} (F_T^T - K) \mathcal{P} (F_T^T) d F_T^T \nonumber \\
    &= \int_{d}^{\infty} (F_t^T e^{-\frac{1}{2} \sigma^2 \tau + \sigma \varepsilon \sqrt{\tau}} - K) \frac{1}{\sqrt{2 \pi}} e^{-\frac{1}{2} \varepsilon^2} d \varepsilon \nonumber \\
    &= F_t^T \int_{d}^{\infty} \frac{1}{\sqrt{2 \pi}} e^{-\frac{1}{2} (\varepsilon^2 - 2 \varepsilon \sigma \sqrt{\tau} + \sigma^2 \tau)} d \varepsilon - K \int_{d}^{\infty} \frac{1}{\sqrt{2 \pi}} e^{-\frac{1}{2} \varepsilon^2} d \varepsilon \nonumber \\
    &= F_t^T \int_{d}^{\infty} \frac{1}{\sqrt{2 \pi}} e^{-\frac{1}{2} (\varepsilon - \sigma \sqrt{\tau})^2} d \varepsilon - K (\Phi(\infty) - \Phi(d)) \nonumber \\
    &= F_t^T \int_{d - \sigma \sqrt{\tau}}^{\infty} \frac{1}{\sqrt{2 \pi}} e^{-\frac{1}{2} \hat{\varepsilon}} d \hat{\varepsilon} - K \Phi (-d) \nonumber \\
    &= F_t^T \Phi(-d + \sigma \sqrt{\tau}) - K \Phi(-d)
\end{align}

Similarly, we can find the future value of put option as the following.
\begin{align}\label{Put option price derivation}
    \mathrm{E} \left[p_T^T \right] &= \int_{0}^{K} (K - F_T^T) * \mathcal{P}(F_T^T) d F_T^T \nonumber \\
    &= \int_{-\infty}^{d} (K - F_t^T e^{-\frac{1}{2} \sigma^2 \tau + \sigma \varepsilon \sqrt{\tau}}) \frac{1}{\sqrt{2 \pi}} e^{-\frac{1}{2} \varepsilon^2} d\varepsilon \nonumber \\
    &= K \int_{-\infty}^{d} \frac{1}{\sqrt{2 \pi}} e^{-\frac{1}{2} \varepsilon^2} d\varepsilon - F_t^T \int_{-\infty}^{d} \frac{1}{\sqrt{2 \pi}} e^{-\frac{1}{2} (\varepsilon^2 - 2 \varepsilon \sigma \sqrt{\tau} + \sigma^2 \tau)} d\varepsilon \nonumber \\
    &= K \int_{-\infty}^{d} \frac{1}{\sqrt{2 \pi}} e^{-\frac{1}{2} \varepsilon^2} d\varepsilon - F_t^T \int_{-\infty}^{d - \sigma \sqrt{\tau}} \frac{1}{\sqrt{2 \pi}} e^{-\frac{1}{2} \hat{\varepsilon}^2} d\hat{\varepsilon} \nonumber \\
    &= K (\Phi(d) - \Phi(-\infty)) - F_t^T (\Phi(d - \sigma \sqrt{\tau}) - \Phi(-\infty)) \nonumber \\
    &= K \Phi(d) - F_t^T \Phi(d - \sigma \sqrt{\tau})
\end{align}

The risk-free zero bond with maturity of $ T $ at current state $ t $ price is $ B_t^T $.
We can get the present value of the options by discounting the forward expected option price.
\begin{align}
    \mathrm{E} \left[c_t^T \right] &= B_t^T \left(F_t^T \Phi \left (\frac{\ln{\frac{F_t^T}{K}} + \frac{1}{2} \sigma^2 \tau}{\sigma \sqrt{\tau}} \right) - K \Phi \left (\frac{\ln{\frac{F_t^T}{K}} - \frac{1}{2} \sigma^2 \tau}{\sigma \sqrt{\tau}} \right)\right) \label{Black76 call price function} \\
    \mathrm{E} \left[p_t^T \right] &= B_t^T \left(K \Phi \left (-\frac{\ln{\frac{F_t^T}{K}} - \frac{1}{2} \sigma^2 \tau}{\sigma \sqrt{\tau}} \right) - F_t^T \Phi \left (-\frac{\ln{\frac{F_t^T}{K}} + \frac{1}{2} \sigma^2 \tau}{\sigma \sqrt{\tau}} \right) \right)\label{Black76 put price function}
\end{align}

Reintroduce $ \eta $ reconcile option pricing formula \ref{Black76 call price function} and \ref{Black76 put price function} into a generic one.
$ V_t^T $ defines the present option price at time $ t $.
\begin{align}\label{Black76 generic price function}
    V_t^T &= B_t^T \eta \left(F_t^T \Phi \left(\eta d_1 \right) - K \Phi \left(\eta d_2 \right) \right) \nonumber \\
    &\text{where, } d_1 = \frac{\ln{\frac{F_t^T}{K}} + \frac{1}{2} \sigma^2 \tau}{\sigma \sqrt{\tau}}; d_2 = d_1 - \sigma \sqrt{\tau}
\end{align}


\section{Greeks}

Greeks are measurements of the instantaneous sensitivity of option prices to the changes of underlying parameters such as forward price, time decay, and volatility.

Several common derivations save later efforts.
\begin{align} \label{relationship between phi(d_1) and phi(d_2)}
    \phi(\eta \cdot x) &= \phi(x) \\
    \phi(d_2) &= \frac{1}{\sqrt{2 \pi}} e^{-\frac{1}{2} (d_1 - \sigma \sqrt{\tau})^2} \nonumber \\
    &= \frac{1}{\sqrt{2 \tau}} e^{-\frac{1}{2} d_1^2} e^{d_1 \sigma \sqrt{\tau} - \frac{1}{2} \sigma \sqrt{\tau}} \nonumber \\
    &= \phi(d_1) e^{\ln{\frac{F}{K}}} \nonumber \\
    &= \phi(d_1) \frac{F}{K} \\
    \eta^2 &= 1
\end{align}

\subsection{Delta} \label{Delta_f derivation}
Delta $ \Delta $ defines the first order derivative of option value against underlying forward.
$ \Delta = \frac{\partial v}{\partial F} $.
\begin{align}
    \frac{\partial d_2}{\partial F_t^T} &= \frac{\partial d_1}{\partial F_t^T} \nonumber \\
    \frac{\partial V_t^T}{\partial F_t^T} &= B_t^T \eta \left( \Phi(\eta d_1) + F_t^T \phi(d_1) \eta \frac{\partial d_1}{\partial F_t^T} - K \phi(d_2) \eta \frac{\partial d_2}{\partial F_t^T} \right) \nonumber \\
    &= B_t^T \eta \Phi(\eta d_1) + B_t^T \eta^2 \left(F_t^T \phi(d_1) \frac{\partial d_1}{\partial F_t^T} - K \phi(d_1) \frac{F_t^T}{K} \frac{\partial d_1}{\partial F_t^T} \right) \nonumber \\
    &= B_t^T \eta \Phi(\eta d_1)
\end{align}

The first derivative of option value against strike $ \Delta_K = \frac{\partial v}{\partial K} $.
\begin{align} \label{Delta_k derivation}
    \frac{\partial d_2}{\partial K} &= \frac{\partial d_1}{\partial K} \nonumber \\
    \frac{\partial V_t^T}{\partial K} &= B_t^T \eta \left (F_t^T \phi(d_1) \eta \frac{\partial d_1}{\partial K} - \Phi(\eta d_2) - K \phi(d_2) \eta \frac{\partial d_2}{K}\right) \nonumber \\
    &= -B_t^T \eta \Phi(\eta d_2) + B_t^T \eta^2 \left(F_t^T \phi(d_1) \frac{\partial d_1}{\partial K} - K \phi(d_1) \frac{F_t^T}{K} \eta \frac{\partial d_1}{\partial K} \right) \nonumber \\
    &= -B_t^T \eta \Phi(\eta d_2)
\end{align}

\subsection{Vega}

Vega $ \nu $ is the first order derivative of option value against the volatility.
$ \nu = \frac{\partial V_t^T}{\partial \sigma} $.
\begin{align} \label{Vega derivation}
    \frac{\partial d_2}{\partial \sigma} &= \frac{\partial d_1}{\partial \sigma} - \sqrt{\tau} \nonumber \\
    \frac{\partial V_t^T}{\partial \sigma} &= B_t^T \eta \left(F_t^T \phi(d_1) \eta \frac{\partial d_1}{\partial \sigma} - K \phi(d_2) \eta \frac{\partial d_2}{\partial \sigma} \right) \nonumber \\
    &= B_t^T \eta^2 \left(F_t^T \phi(d_1) \frac{\partial d_1}{\partial \sigma} - K \phi(d_1) \frac{F_t^T}{K} (\frac{\partial d_1}{\partial \sigma} - \sqrt{\tau}) \right) \nonumber \\
    &= B_t^T F_t^T \phi(d_1) \sqrt{\tau}
\end{align}

\subsection{Theta}

Theta $ \Theta $ defines the first order derivative of option value against the current time $ t $.
$ \Theta = \frac{\partial V_t^T}{\partial t} = -\frac{\partial V_t^T}{\partial \tau}$.
We do not break forward $ F_t^T $ as a function of time to get $ \Theta_f $.
\begin{align} \label{Theta_f derivation}
    \frac{\partial B_t^T}{\partial \tau} &= -r B_t^T \nonumber \\
    \frac{\partial d_2}{\partial \tau} &= \frac{\partial d_1}{\partial \tau} - \frac{\sigma}{2 \sqrt{\tau}} \nonumber \\
    \frac{\partial V_t^T}{\partial \tau} &= -r B_t^T \eta \left(F_t^T \Phi(\eta d_1) - K \Phi(\eta d_2) \right) \nonumber \\
    &\quad \quad + B_t^T \eta \left(F_t^T \phi(d_1) \eta \frac{\partial d_1}{\partial \tau} - K \phi(d_2) \eta \frac{\partial d_2}{\partial \tau}\right) \nonumber \\
    &= -r B_t^T \eta \left(F_t^T \Phi(\eta d_1) - K \Phi(\eta d_2) \right) \nonumber \\
    &\quad \quad + B_t^T \eta^2 \left(F_t^T \phi(d_1) \frac{\partial d_1}{\partial \tau} - K \phi(d_1) \frac{F_t^T}{K} \left(\frac{\partial d_1}{\partial \tau} - \frac{\sigma}{2 \sqrt{\tau}}\right)\right) \nonumber \\
    &= B_t^T F_t^T \phi(d_1) \frac{\sigma}{2 \sqrt{\tau}} - r B_t^T \eta \left(F_t^T \Phi(\eta d_1) - K \Phi(\eta d_2) \right) \nonumber \\
    \frac{\partial V_t^T}{\partial t} &= -B_t^T F_t^T \phi(d_1) \frac{\sigma}{2 \sqrt{\tau}} + r B_t^T \eta \left(F_t^T \Phi(\eta d_1) - K \Phi(\eta d_2) \right)
\end{align}

Another view of the forward based on spot is modeled as $ F_t^T = S_t e^{(r - q) \tau} $ hence $ \Theta_s $ can be derived slightly different from above \ref{Theta_f derivation}.
\begin{align} \label{Theta_s derivation}
    \frac{\partial F_t^T}{\partial \tau} &= (r - q) F_t^T \nonumber \\
    \Theta_s &= -B_t^T F_t^T \phi(d_1) \frac{\sigma}{2 \sqrt{\tau}} + B_t^T \eta \left(q F_t^T \Phi(\eta d_1) - r K \Phi(\eta d_2) \right)
\end{align}

\subsection{Rho}

Rho $ \rho $ defines the first order derivative of option value against the risk free rate $ r $.
$ \rho = \frac{\partial V_t^T}{\partial r} $.
Initially we do not break the forward $ F_t^T $ as a function of $ r $ to get  to get $ \rho_f $
\begin{align} \label{rho_f derivation}
    \frac{\partial B_t^T}{r} &= -\tau B_t^T \nonumber \\
    \frac{\partial d_2}{\partial r} &= \frac{\partial d_1}{\partial r} \nonumber \\
    \frac{\partial V_t^T}{\partial r} &= -\tau B_t^T \eta \left(F_t^T \Phi(\eta d_1) - K \Phi(\eta d_2)\right) \nonumber \\
    &\quad \quad + B_t^T \eta \left(F_t^T \phi(d_1) \eta \frac{\partial d_1}{\partial r} - K \phi(d_2) \eta \frac{\partial d_2}{\partial r}\right) \nonumber \\
    &= -\tau B_t^T \eta \left(F_t^T \Phi(\eta d_1) - K \Phi(\eta d_2)\right) \nonumber \\
    &\quad \quad + B_t^T \eta^2 \left(F_t^T \phi(d_1) \frac{\partial d_1}{\partial r} - K \phi(d_1) \frac{F_t^T}{K} \frac{\partial d_1}{\partial r}\right) \nonumber \\
    &= -\tau B_t^T \eta \left(F_t^T \Phi(\eta d_1) - K \Phi(\eta d_2)\right)
\end{align}

From the spot market of view, forward $ F_t^T $ is also a function of risk free rate $ r $ such that $ F_t^T = S_t e^{(r - q) \tau} $.
Accordingly, the $ \rho_s $ is derived as the following.
\begin{align} \label{rho_s derivation}
    \frac{\partial F_t^T}{\partial r} &= \tau F_t^T \nonumber \\
    \rho_s &= -\tau B_t^T \eta \left(F_t^T \Phi(\eta d_1) - K \Phi(\eta d_2) \right) \nonumber \\
    &\quad \quad + B_t^T \eta \left(\tau F_t^T \Phi(\eta d_2) + F_t^T \phi(d_1) \eta \frac{\partial d_1}{\partial r} - K \phi(d_2) \eta \frac{\partial d_2}{\partial r} \right) \nonumber \\
    &= \tau B_t^T \eta K \Phi(\eta d_2) + B_t^T \eta^2 \left(F_t^T \phi(d_1) \frac{\partial d_1}{\partial r} - K \phi(d_1) \frac{F_t^T}{K} \frac{\partial d_1}{\partial r} \right) \nonumber \\
    &= \tau B_t^T \eta K \Phi(\eta d_2)
\end{align}

\subsection{Gamma}

Gamma $ \Gamma $ is the second order derivative of option price against underlying forward $ F_t^T $.
Equivalently, Gamma is the first order derivative of Delta against forward. 
\begin{align} \label{Gamma_F derivation}
    \frac{\partial d_1}{\partial F_t^T} &= \frac{1}{\sigma \sqrt{\tau}} \frac{K}{F_t^T} \frac{1}{K} \nonumber \\
    &= \frac{1}{F_t^T \sigma \sqrt{\tau}} \nonumber \\
    \frac{\partial^2 V_t^T}{{\partial F_t^T}^2} &= \frac{\partial \Delta}{\partial F_t^T} \nonumber \\
    &= B_t^T \eta^2 \phi(d_1) \frac{\partial d_1}{\partial F_t^T} \nonumber \\
    &= \frac{B_t^T \phi(d_1)}{F_t^T \sigma \sqrt{\tau}}
\end{align}

The second order derivative of option price against strike is $ \Gamma_K $.
Furthermore from the model point of view, $ \Gamma_K $ is actually the expected probability density function of the $ F_T^T $.
\begin{align} \label{Gamma_K derivation}
    \frac{\partial d_2}{\partial K} &= \frac{K}{F_t^T} \left(-\frac{K}{F_t^T} \frac{F_t^T}{K^2} \right) \nonumber \\
    &= \frac{1}{K \sigma \sqrt{\tau}} \nonumber \\
    \frac{\partial^2 V_t^T}{{\partial K}^2} &= \frac{\partial \Delta_K}{\partial K} \nonumber \\
    &= B_t^T \eta^2 \phi(d_2) \frac{\partial d_2}{\partial K} \nonumber \\
    &= \frac{B_t^T \phi(d_2)}{K \sigma \sqrt{\tau}}
\end{align}

\subsection{Vomma}

Vomma $ \frac{\partial^2 V_t^T}{\partial \sigma^2} $ is the second derivative of option price against the volatility $ \sigma $ or the first derivative of vega against volatility $ \frac{\partial \nu}{\partial \sigma} $.
\begin{align}
    \frac{\partial^2 V_t^T}{\partial \sigma^2} &= B_t^T F \phi(d_1) (-d_1) \frac{\partial d_1}{\partial \sigma} \sqrt{\tau} \nonumber \\
    &= -\frac{\partial V_t^T}{\partial \sigma} d_1 \left(-\frac{\ln{\frac{F}{K}}}{\sigma^2 \sqrt{\tau}} + \frac{1}{2} \sqrt{\tau}\right) \nonumber \\
    &= \frac{\partial V_t^T}{\partial \sigma} \frac{d_1 d_2}{\sigma}
\end{align}

\subsection{Vanna}

Vanna $ \frac{\partial^2 V_t^T}{\partial F \partial \sigma} $ is the second order derivative of option price once with respect to underlying $ F $ and once against volatility $ \sigma $.
It can reached as a nested first derivative through either Delta with respect to sigma $ \frac{\partial \Delta}{\partial \sigma} $ or Vega against forward $ \frac{\partial \nu}{\partial F} $.
\begin{align}
    \frac{\partial^2 V_t^T}{\partial F \partial \sigma} &= \eta B_t^T \phi(d_1) \eta \frac{\partial d_1}{\partial \sigma} \nonumber \\
    &= B_t^T \phi(d_1) \left(-\frac{\ln{\frac{F}{K}}}{\sigma^2 \sqrt{\tau}} + \frac{1}{2} \sqrt{\tau}\right) \nonumber \\
    &= -B_t^T \phi(d_1) \frac{d_2}{\sigma} \nonumber \\
    &= -\frac{\partial V_t^T}{\partial \sigma} \frac{d_2}{F \sigma \sqrt{\tau}}
\end{align}

Similarly, Vanna with respect to strike $ \frac{\partial^2 V_t^T}{\partial K \partial \sigma} $ is applied the follows.
\begin{align}
    \frac{\partial^2 V_t^T}{\partial K \partial \sigma} &= -\eta B_t^T \eta \phi(d_2) \frac{\partial d_2}{\partial \sigma} \nonumber \\
    &= -B_t^T \phi(d_2) \left(-\frac{\ln{\frac{F}{K}}}{\sigma^2 \sqrt{\tau}} - \frac{1}{2} \sqrt{\tau} \right) \nonumber \\
    &= B_t^T \phi(d_1) \frac{F}{K} \frac{d_1}{\sigma} \nonumber \\
    &= \frac{\partial V_t^T}{\partial \sigma} \frac{d_1}{K \sigma \sqrt{\tau}}
\end{align}


\section{Partial Differential Equations}

\subsection{Backward PDE}
As mention previously, Black76 model bypasses the drift term and benefits us to reach a cleaner PDE.
\begin{equation} \label{Black76 Backward PDE}
    \frac{\partial V_t^T}{\partial t} + \frac{1}{2} {\left(F_t^T \right)}^2 \sigma^2 \frac{\partial^2 V_t^T}{{\partial F_t^T}^2} - r V_t^T = 0
\end{equation}

A few approaches exists to derive formula \ref{Black76 Backward PDE}.
We provide an relatively intuitive way to derive through delta hedging method as the following.

\subsection{Forward PDE}
Duprie (94) 
\begin{equation} \label{Black76 Forward PDE}
    \frac{\partial V_t^T}{\partial T} - \frac{1}{2} K^2 \sigma^2 \frac{\partial^2 V_t^T}{\partial K} + q V_t^T = 0
\end{equation}



\section{Relationship with Black-Scholes-Merton}

In risk-neutral world, the no-arbitrage price of a forward at current state $ t $ mush equals to $ S_t e^{(r - q) \tau} $ where $ r $ is the risk-free rate and $ q $ is the carry rate of the underlying stock. 


\end{document} 