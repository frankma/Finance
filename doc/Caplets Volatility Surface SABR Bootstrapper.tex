\documentclass{article}
\usepackage[utf8]{inputenc}
\usepackage{amsmath}

\title{Caplets Volatility Surface SABR Bootstrapper}
\author{Frank Ma}
\date{June 2015}

\begin{document}

\maketitle

\begin{abstract}

This paper describes a method to bootstrap cap quotes and calibrate a parametrized model of caplet volatility surface.
The general idea is to calibrate while bootstrapping and reduce the interpolation on cap volatilities or prices to the minimum.
Two fitting goals are under scrutiny: the repricing of calibration instrument and smoothness of caplet volatility surface.

\end{abstract}

\section{Introduction}

Caplets and floorlets are not directly traded in the market.
They must be implied from caps and floors which are actively traded standardized instruments.
Caps and floors are quoted in either flat black volatility or premium.
Two dimensions define caps and floors: expiry and strike.
The expiry typically ranges from 1 Year to 15 Years while the strikes are a set of absolute rates along with ATM straddles with floating term structure.
Any given cap and floor consist of multiple consecutive caplets and floorlets, each is a vanilla option on Libor rate.
\begin{align}
    {payoff}_{caplet}\left(i - 1, i, K\right) &= \tau\left(i - 1, i\right) \left(Libor\left(i - 1, i|{i - 1}\right) - K\right)^{+} \\
    P_{caplet} \left(i-1, i, K, \sigma_i^{caplet}\right) &= B{(0, i)} \mathrm{E}^{\mathrm{Q}} \left[{(L(i - 1, i) - K)}^{+}\right] \\
    P_{cap}\left(0, T, K, \sigma_T^{cap}\right) &= \sum_{i=1}^T{P_{caplet}\left(i - 1, i, K, \sigma_i^{caplet}\right)}
\end{align}

When bootstrapping caplet volatility surfaces out of cap quotes, two major challenges arises.
Firstly, from one cap expiry to the next, multiple caplets exists while only one incremental cap price is known.
To solve such an over-specified problem, assumptions are required to generate unique solution.
Secondly, plainly bootstrap ATM straddles is not applicable as the swap rates vary across different expiries.


\section{Methodologies}

In this section, we start from presenting the commonly used term structure bootstrapping method and corresponding issues on constructing the caplet volatility surface.
Then a global bootstrapping method is introduced to overcome the problems from the term structure bootstrapping method.
To conclude this section, we will overview the pro and cons of the two methods.

\subsection{Term structure bootstrapping method}

Term structure bootstrapping method is based on a straightforward idea: even the number of unknown caplets to the number of known caps by either placing a local interpolator on cap volatility term structure or assuming a specific term structure of caplet volatilities between two cap expiries.
In practice, the method branches to many varieties and are widely used because it is straightforward to comprehend, easy to implement and fast to solve.
Most practitioners prefer to implement local interpolations such as linear, spline or piece wise constant on cap flat volatilities.

Under this method, caps are broken down by strikes and caplet are bootstrapped on interpolated cap vols.
The direct benefit is almost surely repricing of each calibration instruments with minimum of errors.
Two notorious issues, however, arise: exclusion of ATM straddles and a seriously bumpy caplet volatility surface.
Skipping ATM quotes means excluding a great portion of high quality data as straddles are usually most frequently traded.
A non-smooth caplet implied volatility surface not only complicates the exotic instrument pricing but also critically creates undesired trading signal noises that a false arbitrage opportunities exists from the vol surface.
In more common situation, practitioners further fit the bootstrapped caplet volatility surface into a parametrized model such as SABR.
Low quality inputs definitely discredit the trustworthy of the later model outputs.

\subsection{SABR based bootstrapping method}

Since we meant to parametrize the caplet volatility surface into SABR model space, the proposed approach adopts SABR model calibration while doing the bootstrapping.
Any unnecessary assumptions of interpolation on either cap volatilities or prices are lifted but the only interpolation is applied to SABR model parameters.
SABR parameters are interpolated from one cap expiry to the next and interim SABR models are created to price the corresponding caplets.
Direct benefits are apparent: no distinguish is ever needed between fixed and float strikes and volatility smile is naturally captured by the model.
In brief, we are proposing a global bootstrapper method.

Since $ \beta $ and $ \rho $ both control the skewness of implied vol surface, we remove the freedom of beta but choose a deterministic value to simplify the calibration work.
Note a lognormal model is formed when setting $ \beta $ to one; 
a normal model is configured when setting $ \beta $ to zero. 
We assume that model parameter interpolation will be applied to the vol of vol $ \nu $, skewness $ \rho $, and ATM vol level $ \alpha $ as the following.
\begin{subequations}
    \begin{align}
        \nu_{t + j} &= \nu_{t} + \frac{\nu_{t + 1} - \nu_{t}}{\tau_{t + 1} - \tau_t} \left(\tau_{t + j} - \tau_t\right) \\
        \rho_{t + j} &= \rho_{t} + \frac{\rho_{t + 1} - \rho_{t}}{\tau_{t + 1} - \tau_t} \left(\tau_{t + j} - \tau_t\right) \\
        \alpha_{t + j} &= \alpha_{t} + \frac{\alpha_{t + 1} - \alpha_{t}}{\tau_{t + 1} - \tau_t} \left(\tau_{t + j} - \tau_t\right) \\
        &where, \quad 0< j < 1 \nonumber
    \end{align}
\end{subequations}

While bootstrapping, ATM cap price at previous period  $ Cap^{ATM_{t}}{(0, t - 1)} $ is not know but can be implied from the fitted models.
Note $ l $ denotes the length of the underlying libor rate.
\begin{align}
    Cap^{ATM_{t}}{(0, t - 1)} = \sum_{i = 0}^{t - 1 - l}{Cplt\left(i, i + l, K^{ATM_{t}}, \sigma_{B}^{SABR}\left(i, K^{ATM_{t}}\right)\right)} \\
    \sum_{j = -1}^{- l}{Cplt(t + j, t + j + l, K^{ATM_{t}})} = Cap^{ATM_{t}}{(0, t)} - Cap^{ATM_{t}}{(0, t - 1)}
\end{align}

For the first cap expiry, as interpolation is not plausible.
Flat extrapolation is applied.

\subsection{Term structure and global bootstrapping method}

Empirical evidences show apparent trade-offs between exact reprice of the cap prices and smoothness of caplet volatility surface with minimum arbitrage opportunities.
Given the importance of caplet volatility surfaces on pricing off-node caps and exotic instruments, we are better off to have a smooth volatility surface while ensuring calibration instrument re-pricing are within an acceptable tolerance.
Furthermore, given that SABR model is not guaranteed to generate an arbitrage-free volatility surface, we need to be cautious on using the calibrated results at the wings of the volatility surface and extrapolation.

\section{SABR Model}

SABR model is a stochastic volatility model capable of capturing the volatility smile.
One primary use of this model is to interpolate volatility surface. 
\begin{subequations} \label{SABR SDEs}
    \begin{align} 
        dF_t &= \alpha_t F_t^{\beta} dW^{F_t}, \quad F_0 = F\left(0, T\right) \label{SABR fwd SDE} \\
        d\alpha_t &= {\nu}_t \alpha_t dW^{\alpha_t}, \quad \alpha_0 = \alpha \label{SABR vol SDE} \\
        dW^{F} dW^{\alpha} &= \rho dt, \quad \rho \in \left[-1, 1\right] \label{SABR rand correl}
    \end{align}
\end{subequations}

Hangan et al \cite{Managing Smile Risk} present an approximation formula for vanilla European option pricings in the SABR model.
The equivalent implied black volatility of an European option with strike $ K $ and expiry $ T $ is approximated by
\begin{subequations} \label{SABR lognormal vol}
    \begin{align}
        \sigma_B\left(F, K\right) &\approx \frac{\alpha}{{\left(FK\right)}^{\frac{1 - \beta}{2}} \left(1 + \frac{{(1 - \beta)}^{2}}{24} \ln^{2}{\frac{F}{K}} + \frac{{(1 - \beta)}^{4}}{1920} \ln^{4}{\frac{F}{K}}\right)} \nonumber \cdot{\frac{z}{x\left(z\right)}} \nonumber \\
        & \cdot{\left(1 + \left(\frac{\alpha^2 {(1 - \beta)}^{2}}{24 {(F K)}^{1 - \beta}} + \frac{\alpha \beta \nu \rho}{4 {(F K)}^{\frac{1 - \beta}{2}}} + \frac{\nu^2 \left(2 - 3 \rho^2\right)}{24} \right) T \right)} \\
        z &= \frac{\nu}{\alpha} {(F K)} ^ {\frac{1 - \beta}{2}} \ln{\frac{F}{K}} \\
        x\left(z\right) &= \ln{\frac{\sqrt{1 - 2 \rho z + z^2} - \rho + z}{1 - \rho}}
    \end{align}
\end{subequations}

Similarly, the normal volatility approximation is provided as the following.
\begin{subequations} \label{SABR normal vol}
    \begin{align}
        \sigma_N\left(F, K\right) &\approx \alpha \left(F K\right)^{\frac{\beta}{2}} \frac{1 + \frac{1}{24} \ln^2{\frac{F}{K} + \frac{1}{1920} \ln^4{\frac{F}{K}}}}{1 + \frac{\left(1 - \beta\right)^2}{24} \ln^2{\frac{F}{K} + \frac{\left(1 - \beta\right)^2}{1920} \ln^4{\frac{F}{K}}}} \cdot{\frac{z}{x\left(z\right)}} \nonumber \\
        & \cdot{\left(1 + \left(-\frac{\alpha^2 \beta \left(2 - \beta\right)}{24 \left(F K\right)^{1 - \beta}} + \frac{\alpha \beta \nu \rho}{4 \left(F K\right)^{\frac{1 - \beta}{2}}} + \frac{\nu^2 \left(2 - 3 \rho^2\right)}{24}  \right) T\right)} \\
        z &= \frac{\nu}{\alpha} \left(F K\right)^{\frac{1 - \beta}{2}} \ln{\frac{F}{K}} \\
        x\left(z\right) &= \ln{\frac{\sqrt{1 - 2 \rho z + z^2} - \rho + z}{1 - \rho}}
    \end{align}
\end{subequations}

Under normal volatility model, the restriction of strictly positive on forward and strikes should be lifted.
However, the approximation of normal volatility in formula \ref{SABR normal vol} might yield imaginary number under the case of negative forward or strike. An intuitive fix is to force beta to zero which reduces the formula \ref{SABR normal vol} to the following.
\begin{subequations} \label{SABR normal vol with zero beta}
    \begin{align}
        \sigma_N\left(F, K\right) &\approx \alpha \cdot \frac{z}{x\left(z\right)} \cdot \left(1 + \frac{\nu^2 \left(2 - 3 \rho^2\right)}{24} T\right) \\
        z &= \frac{\nu}{\alpha} \left(F - K\right) \\
        x\left(z\right) &= \ln{\frac{\sqrt{1 - 2 \rho z + z^2} - \rho + z}{1 - \rho}}
    \end{align}
\end{subequations}


\section{Calibration}

When calibrating SABR model, we usually assume a specific $ \beta $ level to start with.
That means the rest three model parameters ($ \nu, \rho, \alpha $) need to be searched.
From the nature of the model, each parameter corresponds to a certain feature of the implied volatility surface.
In general, $ \nu $ controls the convexity of the surface;
$ \rho $ changes the skewness;
and $ \alpha $ correspond to the at-the-money volatility level.

Note, $ \alpha $ term has an analytical form, when at-the-money implied volatility is known and $ \beta $ and $ \rho $ are given.
For the ATM implied volatility, SABR black volatility approximation formula \ref{SABR normal vol} will be reduced to the following.
\begin{align}
    \sigma_B\left(F, F\right) &\approx term1 \cdot term2 \cdot term3 \\
    term1 &= \frac{\alpha}{F^{1 - \beta}} \nonumber \\
    term2 &= \lim_{z \to 0} \frac{z}{x(z)} \nonumber \\
    &= \lim_{z \to 0} \frac{1}{\frac{1 - \rho}{\sqrt{1 - 2 \rho z + z^2} - \rho + z} \cdot \frac{\frac{1}{2} {(1 - 2 \rho z + z^2)}^{-\frac{1}{2}} (- 2 \rho + 2 z) + 1}{1 - \rho}} \nonumber \\
    &= \lim_{z \to 0} \sqrt{1 - 2 \rho z + z^2} \nonumber \\
    &= 1 \nonumber \\
    term3 &= 1 + \left(\frac{\alpha^2 {(1 - \beta)}^2}{24 F^{2 - 2\beta}} + \frac{\alpha \beta \nu \rho}{4 F^{1 - \beta}} + \frac{\nu^2 \left(2 - 3 \rho^2\right)}{24} \right) T \nonumber
\end{align}

Reorganize to a cubic polynomial equation on $ \alpha $ as the following.
\begin{equation} \label{alpha cubic polynomial}
    \frac{{(1 - \beta)}^2 T}{24 F^{3 - 3 \beta}} \alpha^3 + \frac{\beta \nu \rho T}{4 F^{2 - 2 \beta}} \alpha^2 + \frac{1 + \frac{\nu^2 \left(2 - 3 \rho^2\right)}{24} T}{F^{1 - \beta}} \alpha - \sigma_B\left(F, F\right) = 0
\end{equation}

The solution of the cubic polynomial equation \ref{alpha cubic polynomial} might range from one to three.
Criteria must be satisfied that $ \alpha $ should be strictly larger than zero.
When multiple greater-than-zero solutions coexists, closest one to the $ \alpha $ approximation will be chosen.
\begin{equation}
    \alpha \approx \frac{\sigma_B(F, F)}{F^{1 - \beta}}
\end{equation}

Similarly, the cubic polynomial equation of normal volatility approximation on $ \alpha $ is as below.
\begin{equation}
    \frac{\left(\left(1 - \beta\right)^2 - 1\right) T}{24 F^{2 - 3 \beta}} \alpha^3 + \frac{\beta \nu \rho T}{4 F^{1 - 2 \beta}} \alpha^2 + \frac{1 +  \frac{\nu^2 \left(2 - 3 \rho^2 T \right)}{24} }{F^{-\beta}} \alpha - \sigma_N\left(F, F\right) = 0
\end{equation}

With the approximation to break the tie.
\begin{equation}
    \alpha \approx \frac{\sigma_N \left(F, F \right)}{F^{-\beta}}
\end{equation}


\begin{thebibliography}{9}
\bibitem{Managing Smile Risk}
Hagan, P., Kumar, D., Lesniewski, A., Woodward, D.
\textit{Managing Smile Risk}
Wilmott Magazine, September, 2002.
\end{thebibliography}

\end{document}