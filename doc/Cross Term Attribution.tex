\documentclass{article}
\usepackage[utf8]{inputenc}
\usepackage{amsmath}
\setlength\parindent{0pt}


\title{Cross Term Attribution}
\author{Frank Ma}
\date{Initial Creation: Nov 2017 \\ Last Update: Nov 2017}

\begin{document}

\maketitle


\section{Introduction}
In theory, one cannot truly decomposes the compound cross term to a linear combination of the contributing risk factors $ x_i $. In practice, however, cross terms are normally broken down and reassigned to contribution factors for simplicity. Multiple approaches are used in practice to allocate cross terms. This paper defines an objective approach to satisfy the needs.


\section{Valuation Decomposition}
Consider an asset price $ V $ as a function of contribution risk factors $ x_i $
\begin{equation}
    V(x_1, x_2, \dots, x_m)
\end{equation}
Taylor's theorem finds the decomposition of changes of asset value to contributing risk factors as the following
\begin{equation}
    \Delta_V = \sum^m{\left\{\frac{\partial V}{\partial x_i} \Delta_{x_i} + \frac{1}{2} \frac{\partial^2 V}{\partial x_i^2} \Delta_{x_i}^2 + \mathcal{O}\left(\Delta_{x_i}^3\right)\right\} + \mathcal{O}\left(\Delta_{x_i} \cdot \Delta_{x_j}\right)}
\end{equation}
Define uni-variate pricing shocks per risk factor as below
\begin{equation}
    \Delta_V^{x_i} = \frac{\partial V}{\partial x_i} \Delta_{x_i} + \frac{1}{2} \frac{\partial^2 V}{\partial x_i^2} \Delta_{x_i}^{2} + \mathcal{O}\left(\Delta_{x_i}^{3}\right)
\end{equation}
Then the cross term $ \mathcal{O}\left(\Delta_{x_i} \cdot \Delta_{x_j}\right) $ denoted as $ \varepsilon $ is backed out as the following
\begin{equation}
    \varepsilon = \Delta_V - \sum^m{\Delta_V^{x_i}}
\end{equation}


\section{Residual Attribution}
The challenge here is to attribute this none-linear term to each contributing factors $ x_i $ through a linear combination. In this study, we manage to find optimal weights $ \omega_i $ proportional to the cross term $ \varepsilon $ contributed by factor $ x_i $. This attribution is justified through a cost function $ \mathcal{F} $. Note this cost function is normalized by factor shocks $ \Delta_{V}^{x_i} $ across all $ n $ valuation scenarios.
\begin{subequations}
    \begin{align}
        \varepsilon_i &= \omega_i \varepsilon \\
        \mathcal{F} &= \sum^n{\sum^m\left(\frac{\varepsilon_i}{\Delta_{V}^{x_i}}\right)^2} \\
        \sum^m{\omega_i} &= 1
    \end{align}
\end{subequations}
The contribution of cross term must be optimized if the total cost is minimized. Noticed this must be a constrained optimization, we embed the weight constrain into the cost function and rearrange the formula to the following.
\begin{equation}
    \mathcal{F} = \sum^{m-1}{\left(\omega_i^2 \sum^n{\left(\frac{\varepsilon}{\Delta_{V}^{x_i}}\right)^2}\right)} + \left(1 - \sum^{m-1}{\omega_i} \right)^2 \sum^n{\left(\frac{\varepsilon}{\Delta_{V}^{x_m}}\right)^2}
\end{equation}
To simplify the above equation, define $ \eta_i $
\begin{equation}
    \eta_i = \sum^n{\left(\frac{\varepsilon}{\Delta_V^{x_i}}\right)^2}
\end{equation}
The cost function is arranged as below
\begin{equation}
    \mathcal{F} = \sum^{m - 1}{\omega_i^2 \eta_i} + \left(1 - \sum^{m - 1}{\omega_i}\right)^2 \eta_m
\end{equation}
To minimize the cost function, we find the first derivatives of $ \mathcal{F} $ against $ \omega_i $ as below.
\begin{equation}
    \frac{\partial \mathcal{F}}{\partial \omega_i} = 2 \omega_i \eta_i - 2 \left(1 - \sum^{m - 1}{\omega_i}\right)\eta_m
\end{equation}
The optimization results of $ \omega_i $ is found through solving a system of formulas where $ \frac{\partial \mathcal{F}}{\partial \omega_i} \equiv 0 $
\begin{equation}
    \omega_i \eta_i + \sum^{m - 1}{\omega_i} \eta_m = \eta_m, \forall \omega_i \in [1, m - 1]
\end{equation}
Concisely, in the matrix expression
\begin{equation}
    \left(\boldsymbol{H} + \eta_m\right) \cdot \boldsymbol{\Omega} = \eta_m \cdot \boldsymbol{1}
\end{equation}
where
\begin{equation}
    \boldsymbol{H} = \left| \begin{matrix} \eta_1 & & & \\ & \eta_2 & & \\ & & \ddots & \\ & & & \eta_{m- 1} \end{matrix} \right |_{(m - 1) \times (m - 1)} \boldsymbol{\Omega} = \left| \begin{matrix} \omega_1 \\ \omega_2 \\ \vdots \\ \omega_{m - 1} \end{matrix} \right|_{(m - 1) \times 1} \boldsymbol{1} = \left| \begin{matrix} 1 \\ 1 \\ \vdots \\ 1 \end{matrix} \right|_{(m - 1) \times 1} \nonumber
\end{equation}
To solve for the optimal weights, we need to invert the diagonal plus a constant matrix $ \boldsymbol{H} + \eta_m $. The Sherman-Morrison formula finds the solution as below
\begin{equation}
    \left( \boldsymbol{H} + \eta_m \right)^{-1} = \boldsymbol{H}^{-1} - \frac{\eta_m \cdot \boldsymbol{H}^{-1} \cdot \boldsymbol{1} \cdot \boldsymbol{1}^T \cdot \boldsymbol{H}^{-1}}{1 + \eta_m \cdot \boldsymbol{1}^{T} \cdot \boldsymbol{H}^{-1} \cdot \boldsymbol{1}}
\end{equation}
Then the weights are in the form
\begin{equation}
    \boldsymbol{\Omega} = \boldsymbol{H}^{-1} \cdot \boldsymbol{1} \cdot \eta_m - \frac{\eta_m}{1 + \eta_m \cdot \boldsymbol{1}^{T} \cdot \boldsymbol{H}^{-1} \cdot \boldsymbol{1}} \cdot \boldsymbol{H}^{-1} \cdot \boldsymbol{1} \cdot \boldsymbol{1}^{T} \cdot \boldsymbol{H}^{-1} \cdot \boldsymbol{1} \cdot \eta_m
\end{equation}
Work out each component separately in details to find the following
\begin{subequations}
    \begin{align}
        \boldsymbol{H}^{-1} \cdot \boldsymbol{1} \cdot \eta_m &= \left| \begin{matrix} \frac{1}{\eta_1} \\ \frac{1}{\eta_2} \\ \vdots \\ \frac{1}{\eta_{m - 1}} \end{matrix}\right| \cdot \eta_m \\
        \frac{\eta_m}{1 + \eta_m \cdot \boldsymbol{1}^{T} \cdot \boldsymbol{H}^{-1} \cdot \boldsymbol{1}} &= \frac{1}{\frac{1}{\eta_m} + \sum^{m - 1}{\frac{1}{\eta_i}}} \nonumber \\
        &= \frac{1}{\sum^{m}{\frac{1}{\eta_i}}} \\
        \boldsymbol{H}^{-1} \cdot \boldsymbol{1} \cdot \boldsymbol{1}^{T} \cdot \boldsymbol{H}^{-1} \cdot \boldsymbol{1} \cdot \eta_m &= \left| \begin{matrix} \frac{1}{\eta_1^2} + \frac{1}{\eta_1 \eta_2} + \dots + \frac{1}{\eta_1 \eta_{m - 1}} \\ \frac{1}{\eta_2 \eta_1} + \frac{1}{\eta_{2}^{2} +} \dots + \frac{1}{\eta_2 \eta_{m - 1}}  \\ \vdots \\ \frac{1}{\eta_{m - 1} \eta_1} + \frac{1}{\eta_{m - 1} \eta_2} + \dots + \frac{1}{\eta_{m - 1}^{2}} \end{matrix} \right| \cdot \eta_m \nonumber \\
        &= \left| \begin{matrix}\frac{1}{\eta_1} \left(\sum^{m - 1}{\frac{1}{\eta_i}}\right) \\ \frac{1}{\eta_2} \left(\sum^{m - 1}{\frac{1}{\eta_i}}\right) \\ \vdots \\ \frac{1}{\eta_{m - 1}} \left(\sum^{m - 1}{\frac{1}{\eta_i}}\right) \end{matrix} \right| \cdot \eta_m \nonumber \\
        &= \left| \begin{matrix} \frac{\eta_m}{\eta_1} \\ \frac{\eta_m}{\eta_2} \\ \vdots \\ \frac{\eta_m}{\eta_{m - 1}} \end{matrix} \right| \cdot \sum^{m - 1}{\frac{1}{\eta_i}} \cdot \eta_m
    \end{align}
\end{subequations}
Exam all components together to work out the weights as the following
\begin{align}
    \boldsymbol{\Omega} &= \left| \begin{matrix} \frac{1}{\eta_1} \\ \frac{1}{\eta_2} \\ \vdots \\ \frac{1}{\eta_{m - 1}} \end{matrix} \right| \cdot \eta_m \cdot \left(1 - \frac{\sum^{m - 1}{\frac{1}{\eta_i}}}{\sum^{m}{\frac{1}{\eta_i}}} \right) \nonumber \\
    &= \left| \begin{matrix} \frac{1}{\eta_1} \\ \frac{1}{\eta_2} \\ \vdots \\ \frac{1}{\eta_{m - 1}} \end{matrix} \right| \cdot \frac{1}{\sum^{m}{\frac{1}{\eta_i}}}
\end{align}
Note for the summation of reciprocals can be expanded as below
\begin{align}
    \sum^{m}{\frac{1}{\eta_i}} &= \frac{\eta_2 \eta_3 \dots \eta_m + \eta_3 \eta_4 \dots \eta_m + \dots + \eta_1 \eta_2 \dots \eta_{m - 1}}{\eta_1 \eta_2 \dots \eta_m} \nonumber \\
     &= \frac{\sum_{i = 1}^{m}{\prod_{j = 1, j \neq i}^{m}{\eta_j}}}{\prod^{m}_{j = 1}{\eta_j}}
\end{align}
To find out the embedded weight $ \omega_m $, we revisit the weight constrain and find
\begin{align}
    \omega_m &= 1 - \frac{\sum^{m - 1}{\frac{1}{\eta_i}}}{\sum^{m}{\frac{1}{\eta_i}}} \nonumber \\
    &= \frac{\prod^{m}_{j \neq m}{\eta_j}}{\sum^{m}_{i=1}{\prod^{m}_{j \neq i}{\eta_j}}}
\end{align}
Observed a generic form of weights in above solution holds for each $ \omega_i $ as 
\begin{equation}
    \omega_i = \frac{\prod^{m}_{j \neq i}{\eta_j}}{\sum^{m}_{i = 1}{\prod^{m}_{j \neq i}{\eta_j}}}
\end{equation}
The full spectrum of weights $ \boldsymbol{\Omega}_{m \times 1} $ is formulated as
\begin{equation}
    \boldsymbol{\Omega}_{m \times 1} = \left| \begin{matrix} \frac{\prod_{j \neq 1}^{m}{\eta_j}}{\sum_{i = 1}^{m}{\prod_{j = 1, j \neq i}^{m}{\eta_j}}} \\ \\ \frac{\prod_{j \neq 2}^{m}{\eta_j}}{\sum_{i = 1}^{m}{\prod_{j = 1, j \neq i}^{m}{\eta_j}}} \\ \vdots \\ \frac{\prod_{j \neq m - 1}^{m}{\eta_j}}{\sum_{i = 1}^{m}{\prod_{j = 1, j \neq i}^{m}{\eta_j}}} \\ \\ \frac{\prod_{j \neq m}^{m}{\eta_j}}{\sum_{i = 1}^{m}{\prod_{j = 1, j \neq i}^{m}{\eta_j}}} \end{matrix} \right|_{m \times 1}
\end{equation}


\section{Computation Complexity}
There are three sections of calculation blocks. The first one is to compute $ \eta $ which requires iteration through all scenarios and factors. The complexity is $ \boldsymbol{O}(m \times n) $. The second work is to find optimal weights $ \Omega $. The work load is $ \boldsymbol{O}(m) $ as the computation actually resembles running products. The last piece is to attribute cross terms per scenario per factor. This section is as straightforward as $ \boldsymbol{O}(n \times m) $. That leaves the overall computation complexity as of $ \boldsymbol{O}(n \times m) $.

\end{document}